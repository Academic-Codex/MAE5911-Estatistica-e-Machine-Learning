\documentclass[a4paper]{article}
\usepackage{student}
\usepackage{graphicx}
\usepackage{caption}
\usepackage[version=4]{mhchem}
\usepackage{tikz}
\usetikzlibrary{shapes.geometric, arrows.meta, positioning, decorations.pathreplacing}
\usepackage{enumitem}
\usepackage[utf8]{inputenc}
\usepackage{amsmath}
\usepackage{booktabs}
\usepackage{float}
\usepackage[portuguese]{babel}
\usepackage[T1]{fontenc}
\usepackage[utf8]{inputenc}
\usepackage[numbers,sort&compress]{natbib} % ou [authoryear]
\usepackage[alf]{abntex2cite} % citação ABNT autor-data
\usepackage{listings}
\usepackage{xcolor}
\usepackage{url}
\lstset{
  basicstyle=\ttfamily\small,
  commentstyle=\color{gray}\itshape,
  keywordstyle=\color{blue},
  stringstyle=\color{orange},
  frame=single,
  backgroundcolor=\color{gray!5},
  breaklines=true,
  captionpos=b
}
% --- Suporte a UTF-8 ---
\usepackage{lmodern}           % fonte moderna compatível com T1

% --- Pacotes para exibir código ---


% --- Define o idioma R com palavras-chave e cores ---
\lstdefinelanguage{R}{
  keywords={function, if, else, for, in, while, repeat, return, library, require, next, break},
  keywordstyle=\color{blue}\bfseries,
  ndkeywords={TRUE, FALSE, NULL, NA, Inf, NaN},
  ndkeywordstyle=\color{magenta}\bfseries,
  comment=[l]{\#},
  commentstyle=\color{gray}\ttfamily,
  stringstyle=\color{orange}\ttfamily,
  morestring=[b]",
  morestring=[b]',
  sensitive=true
}

% --- Mapeamento para UTF-8 (acentos, ç, etc.) ---
\lstset{
  literate=
    {á}{{\'a}}1 {ã}{{\~a}}1 {â}{{\^a}}1 {à}{{\`a}}1
    {é}{{\'e}}1 {ê}{{\^e}}1
    {í}{{\'i}}1
    {ó}{{\'o}}1 {õ}{{\~o}}1 {ô}{{\^o}}1
    {ú}{{\'u}}1 {ü}{{\"u}}1
    {ç}{{\c{c}}}1
    {Á}{{\'A}}1 {Ã}{{\~A}}1 {Â}{{\^A}}1
    {É}{{\'E}}1 {Ê}{{\^E}}1
    {Í}{{\'I}}1
    {Ó}{{\'O}}1 {Õ}{{\~O}}1 {Ô}{{\^O}}1
    {Ú}{{\'U}}1 {Ü}{{\"U}}1
    {Ç}{{\c{C}}}1,
  basicstyle=\ttfamily\small,
  backgroundcolor=\color{gray!5},
  frame=single,
  breaklines=true,
  captionpos=b,
  columns=flexible,
  showstringspaces=false,
  keepspaces=true,
  numbers=left,
  numberstyle=\tiny\color{gray},
  stepnumber=1,
  numbersep=6pt
}
\pagestyle{plain}

\tikzstyle{arrow} = [thick,->,>=stealth]

% Definindo o estilo de destaque com linhas pontilhadas
\tikzstyle{highlight} = [draw, dashed, thick, rectangle, rounded corners, inner sep=0.2cm, orange]


\tikzstyle{startstop} = [
    rectangle, rounded corners, minimum width=0.5cm,
    text centered, draw=black, fill=blue!10, font=\small
]
\tikzstyle{startstop_S} = [
    rectangle, rounded corners, minimum width=0.5cm, minimum height=0.8cm,
    text centered, draw=black, fill=green!30, font=\small
]
\tikzstyle{decision} = [
    diamond, aspect=2, draw=black, fill=orange!15, align=center,
    text centered, inner sep=0pt, font=\small
]
\tikzstyle{decision_S} = [
    diamond, aspect=2, draw=black, fill=orange!30, align=center,
    text centered, inner sep=0pt, font=\small
]
\tikzstyle{arrow} = [thick,->,>=stealth]



% Metadata
\date{\today}
\setmodule{MAE5911/IME: Fundamentos de Estatística e Machine Learning. \\ Prof.: Alexandre Galvão Patriota} 
\setterm{2o. semestre, 2025}

%-------------------------------%
% Other details
% TODO: Fill these
%-------------------------------%
\title{Prova - 04/11}
\setmembername{Nara Avila Moraes}  % Fill group member names
\setmemberuid{5716734}  % Fill group member uids (same order)

%-------------------------------%
% Add / Delete commands and packages
% TODO: Add / Delete here as you need
%-------------------------------%
\usepackage{amsmath,amssymb,bm}

\newcommand{\KL}{\mathrm{KL}}
\newcommand{\R}{\mathbb{R}}
\newcommand{\E}{\mathbb{E}}
\newcommand{\T}{\top}

\newcommand{\expdist}[2]{%
        \normalfont{\textsc{Exp}}(#1, #2)%
    }
\newcommand{\expparam}{\bm \lambda}
\newcommand{\Expparam}{\bm \Lambda}
\newcommand{\natparam}{\bm \eta}
\newcommand{\Natparam}{\bm H}
\newcommand{\sufstat}{\bm u}

% Main document
\begin{document}
    % Add header
    \header{}

\textbf{Questão 01:} Defina formalmente o valor-p (\textbf{aproximado}) para testar uma hipótese geral e discuta os problemas de interpretação usuais ao utilizar a sua versão condicional à hipótese nula. Apresente um exemplo numérico e simulações de Monte Carlo, usando um modelo de regressão Poisson, para ilustrar a sua aplicação. \\[0.5cm]
\textbf{Questão 02:} Apresente um Teorema da Aproximação Universal que generalize o Teorema de Funahashi, discuta as diferenças entre os resultados por meio de exemplos de redes neurais. \\[0.5cm]
\textbf{Questão 03:} Proponha uma função de estimação robusta para estimar os parâmetros de uma regressão Binomial, $Y|X=x \sim \mathrm{Bin}(m,\mu_\theta(x))$, em que $\mu_\theta(\cdot)$ é uma função com imagem em $[0,1]$ e $\theta$ é o vetor de parâmetros. Apresente um exemplo numérico e uma simulação de Monte Carlo, como feito em sala, para ilustrar os resultados. Perturbe a distribuição dos dados e mostre que a sua proposta é de fato robusta contra essa perturbação. \\[0.5cm]
\begin{answer}[Questão 01]
O conceito do p-valor, tal como proposto originalmente por \textit{Fisher} (1925), é uma medida \textbf{condicional sob} $H_0$, definida por
\[
\text{valor-p}(H_0, z_n) = \sup_{\theta \in \Theta_0} P_\theta^{(n)}\big(T_{H_0}(Z_n) \ge T_{H_0}(z_n)\big),
\]
isto é, a probabilidade de se observar uma estatística tão extrema quanto a obtida, assumindo $H_0$ verdadeira.
O valor-p foi concebido por Fisher como um \textit{instrumento de contraste} entre o dado e hipótese, mas não como uma medida de verdade.
Seu objetivo era expressar o grau de incompatibilidade empírica entre o fenômeno observado e as consequências lógicas de $H_0$.

Posteriormente, a escola \textit{Neyman--Pearson} reinterpretou esse conceito dentro de uma estrutura de decisão.
Nessa transição, o valor-p perdeu seu caráter exploratório e passou a ser tratado como um critério binário, externo e supostamente não condicionado a $H_0$, capaz de expressar a decisão: ``rejeita'' ou ``não rejeita'' $H_0$.
Esta leitura gerou um equívoco lógico: o valor-p, que é definido dentro do universo em que $H_0$ é verdadeira,
passou a ser interpretado como se pudesse mensurar a probabilidade de $H_0$ ser verdadeira.

O \textbf{valor-p assintótico}, que é o valor usualmente utilizado, na ausência da distribuição exata de $T_{H_0}$, é definido por:
\[
p^{(a)}(H_0, z_n) = \sup_{\theta \in \Theta_0} P_\theta(W_\theta \ge T_{H_0}(z_n)),
\]
em que $W_\theta$ representa a distribuição-limite de $T_{H_0}$ quando $n \to \infty$.

Confirmando essa natureza epistêmica, o valor-p assintótico nem sempre converge para o valor-p exato, isto porque, estando definido condicionalmente a $H_0$, pode não ser sensível a falhas estruturais do modelo, 
como a ausência de independência entre as observações, a presença de hipóteses condicionais aninhadas, ou a violação das condições de regularidade que garantem a validade assintótica — 
como homocedasticidade. 

Por isso, o valor-p deve ser entendido apenas como originalmente proposto — 
uma \textbf{medida de discrepância entre o real e o teórico} — 
que opera em um único sentido: como evidência contra a hipótese nula, mas nunca para confirmá-la.

Em termos das $\sigma$-álgebras envolvidas, a construção inferencial pode ser representada como uma sequência de aplicações mensuráveis, onde cada espaço de probabilidade possui a sua própria $\sigma$-álgebra, refletindo níveis distintos de abstração:
\[
(\Omega_n, \mathcal{F}_n, \mathbb{P}_n)
\;\xrightarrow{Z_n}\;
(\mathcal{Z}_n, \mathcal{B}_{Z_n}, \mathbb{P}_{Z_n})
\;\xrightarrow{T}\;
(\mathcal{T}_n, \mathcal{B}_{T_n}, \mathbb{P}_{T_n}).
\]

\begin{itemize}
    \item $\mathcal{F}_n$ — descreve o universo empírico de incertezas, isto é, os eventos do mundo observável;
    \item $\mathcal{B}_{Z_n}$ — corresponde ao espaço amostral modelado, onde as variáveis aleatórias $Z_n$ são definidas sob $\mathbb{P}_{Z_n}$;
    \item $\mathcal{B}_{T_n}$ — é a $\sigma$-álgebra induzida pela estatística $T$, onde vivem as distribuições teóricas dos testes e, em particular, o valor-p.
\end{itemize}

O valor-p pertence ao universo lógico em que a hipótese nula $H_0$ é assumida como verdadeira.
Ele não pertence à $\sigma$-álgebra empírica $\mathcal{F}_n$, e portanto não pode ser interpretado
como uma probabilidade sobre o mundo real ou sobre a veracidade de $H_0$.
Tratá-lo dessa forma constitui um erro de violação da $\sigma$-álgebra — 
um deslize epistemológico em que se tenta extrair informação empírica de uma construção em sua definição epistêmica condicional e autoreferente.

\subsection*{Implementações}

%%%%%%%%%%%%%%%%%%%%%%%%%%%%%%%%%%%%%%%%%%%%%%%%%%%%%%%%%%%%%%%%%%%%%%%%%%%%%%%%%%%%%%%%%%%%%%%%%%%%%%%%%%%%%%%%%%%%%%
%%%%%%%%%%%%%%%%%%%%%%%%%%%%%%%%%%%%%%%%%%%%%%%%%%%%%%%%%%%%%%%%%%%%%%%%%%%%%%%%%%%%%%%%%%%%%%%%%%%%%%%%%%%%%%%%%%%%%%
% Numerico
O exemplo numérico apresentado na Listing~\ref{lst:poisson-numerico} tem o objetivo de ilustrar o comportamento do valor-$p$ sob diferentes hipóteses nulas, mantendo fixos os dados observados. 
O experimento baseia-se em um modelo de regressão de Poisson definido por
\[
Y_i \sim \text{Poisson}(\lambda_i), \quad \text{com} \quad \log(\lambda_i) = \theta_0 + \theta x_i,
\]
onde $\theta_0 = 1$ e o valor verdadeiro do parâmetro é $\theta = 0{,}20$. 
Uma única amostra de tamanho $n = 100$ foi gerada segundo esse modelo, representando o conjunto de dados ``observados'' em um mundo empírico fixo.

A partir dessa amostra, foram testadas três hipóteses nulas distintas sobre o mesmo conjunto de dados:
\[
H_{0}^{(A)} = 0{,}20, \qquad
H_{0}^{(B)} = 0{,}28, \qquad
H_{0}^{(C)} = 0{,}60.
\]
Cada hipótese foi avaliada por meio da estatística de Wald,
\[
Z = \frac{\hat{\theta} - \theta_{H_0}}{\operatorname{se}(\hat{\theta})},
\]
e o valor-$p$ correspondente foi calculado de forma:
\[
p = P(Z \le z_{\text{obs}}),
\]
correspondendo à hipótese alternativa $H_1 : \theta < \theta_{H_0}$.

Os resultados obtidos encontram-se na Tabela~\ref{tab:resultado_simulacao}. 
Observa-se que, quando a hipótese nula coincide com o valor real do parâmetro ($\theta$), o valor-$p$ é elevado e a hipótese não é rejeitada. 
À medida que o valor hipotético se afasta de $\theta$, o valor-$p$ decresce rapidamente, levando à rejeição de $H_0$ nos casos de maior discrepância ($H_{0}^{(C)}$). 
Essa simulação evidencia o papel do valor-$p$ como medida de compatibilidade entre modelo e dado: ele quantifica o grau de discrepância entre a hipótese estatística e a estrutura empírica observada.
\begin{table}[H]
\centering
\caption{Resultados do cálculo do valor-$p$ para diferentes hipóteses nulas sob o modelo de regressão de Poisson.}
\label{tab:resultado_simulacao}
\begin{tabular}{lccccc}
\toprule
\textbf{Cenário} & \textbf{Hipótese nula} & \textbf{Valor-$p$} & \textbf{Decisão} & \textbf{$\hat{\theta}$} & \textbf{$\theta$} \\
\midrule
Cenário A ($H_{0}^{(A)}$ exato) & $\theta_{H_0} = 0{,}20$ & 0{,}787386 & Não rejeita $H_0$ & 0{,}254359 & 0{,}20 \\
Cenário B ($H_{0}^{(B)}$ com pequeno desvio) & $\theta_{H_0} = 0{,}28$ & 0{,}353412 & Não rejeita $H_0$ & 0{,}254359 & 0{,}20 \\
Cenário C ($H_{0}^{(C)}$ com grande desvio) & $\theta_{H_0} = 0{,}60$ & 0{,}000000 & Rejeita $H_0$ & 0{,}254359 & 0{,}20 \\
\bottomrule
\end{tabular}
\end{table}

%%%%%%%%%%%%%%%%%%%%%%%%%%%%%%%%%%%%%%%%%%%%%%%%%%%%%%%%%%%%%%%%%%%%%%%%%%%%%%%%%%%%%%%%%%%%%%%%%%%%%%%%%%%%%%%%%%%%%%
%%%%%%%%%%%%%%%%%%%%%%%%%%%%%%%%%%%%%%%%%%%%%%%%%%%%%%%%%%%%%%%%%%%%%%%%%%%%%%%%%%%%%%%%%%%%%%%%%%%%%%%%%%%%%%%%%%%%%%
% Monte Carlo
O experimento de Monte Carlo tem como objetivo avaliar, de forma empírica, 
o comportamento assintótico do valor-$p$ quando múltiplas amostras são geradas a partir 
de um mesmo modelo de regressão de Poisson. 
Considera-se o modelo

\[
Y_i \sim \text{Poisson}(\lambda_i), 
\quad \text{com} \quad 
\log(\lambda_i) = \theta_0 + \theta x_i,
\]

onde o intercepto $\theta_0 = 1$ e o valor verdadeiro do parâmetro é $\theta = 0{,}20$. 
Em cada replicação, é gerada uma amostra de tamanho $n = 100$ com covariáveis 
$x_i \sim \text{Unif}(0,2)$ e respostas $Y_i$ conforme o modelo acima.
O processo é repetido $M = 10{,}000$ vezes, mantendo $\theta$ fixo.

Para cada amostra, ajusta-se um modelo de regressão de Poisson e testam-se 
três hipóteses nulas distintas sobre o mesmo conjunto de dados:

\[
H_{0}^{(A)} = 0{,}20, 
\qquad
H_{0}^{(B)} = 0{,}28, 
\qquad
H_{0}^{(C)} = 0{,}60.
\]

Cada hipótese é avaliada pela estatística de Wald,

\[
Z = \frac{\hat{\theta} - \theta_{H_0}}{\operatorname{se}(\hat{\theta})},
\]

e o valor-$p$ correspondente é calculado da seguinte forma, 
considerando o teste contra a alternativa $H_1 : \theta < \theta_{H_0}$:

\[
p = P(Z \le z_{\text{obs}}).
\]


A partir das $M$ replicações, calcula-se a proporção de rejeição de cada hipótese, 
a média e a mediana dos valores-$p$, bem como a média dos estimadores $\hat{\theta}$. 
Os resultados são apresentados na Tabela~\ref{tab:montecarlo}.

Observa-se que, quando $H_{0}^{(A)}$ coincide com o valor verdadeiro do parâmetro, 
a proporção de rejeição aproxima-se do nível $\alpha = 0{,}05$, 
e os valores-$p$ seguem aproximadamente uma distribuição uniforme. 
À medida que as hipóteses nulas se deslocam para valores superiores ao verdadeiro 
— como em $H_{0}^{(B)}$ e $H_{0}^{(C)}$ — o teste unicaudal à esquerda 
ganha poder, e os valores-$p$ tornam-se progressivamente menores. 
Nos casos de grande discrepância ($H_{0}^{(C)}$), a rejeição de $H_0$ 
é praticamente certa, refletindo a coerência do teste de Wald 
com o comportamento teórico esperado sob o modelo de Poisson.

\begin{table}[H]
\centering
\caption{Resultados da simulação de Monte Carlo para o teste de Wald em regressão de Poisson. 
Cada linha apresenta os resultados obtidos para uma hipótese nula distinta, com $M = 10{,}000$ replicações e $n = 100$.}
\label{tab:montecarlo}
\begin{tabular}{lccccccc}
\toprule
\textbf{Cenário} & \textbf{H0} & \textbf{\% de rejeição} & \textbf{Média($p$)} & \textbf{Mediana($p$)} & \textbf{$p_{0{,}05}$} & \textbf{$p_{0{,}95}$} & \textbf{$\bar{\hat{\theta}}$ / $\theta$} \\
\midrule
Cenário A 
& $H_{0}^{(A)} = 0{,}20$ 
& 0{,}053700 
& 0{,}494656 
& 0{,}488848 
& 0{,}045770 
& 0{,}950321 
& 0{,}1983 / 0{,}2 \\

Cenário B 
& $H_{0}^{(B)} = 0{,}28$ 
& 0{,}204100 
& 0{,}282815 
& 0{,}204148 
& 0{,}006543 
& 0{,}804670 
& 0{,}1983 / 0{,}2 \\

Cenário C 
& $H_{0}^{(C)} = 0{,}60$ 
& 0{,}989600 
& 0{,}002594 
& 0{,}000030 
& 0{,}000000 
& 0{,}010802 
& 0{,}1983 / 0{,}2 \\
\bottomrule
\end{tabular}
\end{table}
\end{answer}

\begin{lstlisting}[language=R, caption={Exemplo numérico - Regressão Poisson}, label={lst:poisson-numerico}]
# ====================================================================
# Regressão Poisson (GLM)
# ====================================================================

set.seed(111)

# Parâmetros fixos
n      = 100
beta0  = 1.0
theta  = 0.20     # valor verdadeiro do parâmetro (inclinação)
alpha  = 0.05

# Covariável e única amostra observada
x       = runif(n, 0, 2)
lambda  = exp(beta0 + theta * x)
y       = rpois(n, lambda)

# Ajuste único do GLM Poisson nos mesmos dados
modelo  = glm(y ~ x, family = poisson)

# ---------------------------------------------------------------
# Função: testa H0: theta = h0
# ---------------------------------------------------------------
testa_glm_theta = function(mod, h0, alpha = 0.05) {
  theta_hat = coef(mod)[2]
  se_hat    = sqrt(vcov(mod)[2, 2])
  Z         = (theta_hat - h0) / se_hat

  # valor-p unicaudal (contra H1: theta > h0)
  p_valor   = pnorm(Z)

  decisao   = ifelse(p_valor < alpha, "Rejeita H0", "Não rejeita H0")

  data.frame(
    hipotese_nula = paste0("H0: theta = ", format(h0, nsmall = 2)),
    valor_p       = p_valor,
    decisao       = decisao,
    theta_hat     = theta_hat,
    theta         = theta
  )
}

# Hipóteses nulas a testar no mesmo conjunto 
H0_A = 0.20   # H0 exato
H0_B = 0.28   # H0 com pouco desvio
H0_C = 0.60   # H0 com muito desvio

resA = testa_glm_theta(modelo, H0_A, alpha)
resB = testa_glm_theta(modelo, H0_B, alpha)
resC = testa_glm_theta(modelo, H0_C, alpha)

# Consolida com rbind 
resumo_glm = rbind(
  "Cenário A (H0 exato)"        = resA,
  "Cenário B (H0 pouco desvio)" = resB,
  "Cenário C (H0 muito desvio)" = resC
)

# -------- Formatação --------
options(scipen = 999)

resumo_fmt           = resumo_glm
resumo_fmt$valor_p   = format(round(resumo_fmt$valor_p,   6), nsmall = 6)
resumo_fmt$theta_hat = format(round(resumo_fmt$theta_hat, 6), nsmall = 6)
resumo_fmt$theta     = format(round(resumo_fmt$theta,     6), nsmall = 6)

# Tabela final (analítica)
noquote(resumo_fmt)
\end{lstlisting}

\begin{lstlisting}[language=R, caption={Exemplo numérico - Regressão Poisson}, label={lst:poisson-numerico}]
set.seed(111)

# -------------------------
# Parâmetros do experimento
# -------------------------
n      = 200        # tamanho da amostra
beta0  = 1.0        # intercepto fixo
theta  = 0.20       # (verdadeiro) = coef. de x, usado para gerar os dados
alpha  = 0.05       # nível
M      = 10000      # repetições de Monte Carlo

# hipóteses nulas a testar (todas no mesmo conjunto de dados por replicação)
H0_A = 0.20   # H0 exato
H0_B = 0.28   # H0 com pouco desvio
H0_C = 0.60   # H0 com grande desvio
H0s  = c(H0_A, H0_B, H0_C)

# -------------------------------------------
# Design fixo: x é gerado 1 vez e mantido fixo
# -------------------------------------------
x = runif(n, 0, 2)

# ---------------------------------------------------------------
# Função: calcula p-valor unicaudal (esquerda) e decisão para b0
# ---------------------------------------------------------------
p_unicaudal_wald = function(mod, b0, alpha) {
  theta_hat = coef(mod)[2]
  se_hat    = sqrt(vcov(mod)[2, 2])
  Z         = (theta_hat - b0) / se_hat
  # unicaudal à esquerda: H1: theta < b0
  p_valor   = pnorm(Z)
  decisao   = ifelse(p_valor < alpha, "Rejeita H0", "Não rejeita H0")
  list(p = p_valor, decisao = decisao, theta_hat = theta_hat)
}

# ---------------------------------------------------------------
# Uma replicação: gera y ~ Poisson(exp(beta0 + theta * x)), ajusta GLM
# e testa H0_A, H0_B, H0_C na mesma amostra
# ---------------------------------------------------------------
uma_replicacao = function() {
  lambda = exp(beta0 + theta * x)
  y      = rpois(n, lambda)
  mod    = glm(y ~ x, family = poisson)
  lapply(H0s, function(b0) p_unicaudal_wald(mod, b0, alpha))
}

# ---------------------------------------------------------------
# Roda M replicações e consolida resultados por cenário
# ---------------------------------------------------------------
# estruturas de acumulação
p_list_A = numeric(M); dec_A = character(M); thA = numeric(M)
p_list_B = numeric(M); dec_B = character(M); thB = numeric(M)
p_list_C = numeric(M); dec_C = character(M); thC = numeric(M)

for (m in 1:M) {
  out = uma_replicacao()
  # A
  p_list_A[m] = out[[1]]$p
  dec_A[m]    = out[[1]]$decisao
  thA[m]      = out[[1]]$theta_hat
  # B
  p_list_B[m] = out[[2]]$p
  dec_B[m]    = out[[2]]$decisao
  thB[m]      = out[[2]]$theta_hat
  # C
  p_list_C[m] = out[[3]]$p
  dec_C[m]    = out[[3]]$decisao
  thC[m]      = out[[3]]$theta_hat
}

# ---------------------------------------------------------------
# Sumários: proporção de rejeição (potência/erro tipo I),
# estatísticas de p e média de theta_hat
# ---------------------------------------------------------------
resumo_mc = data.frame(
  hipotese_nula = c(
    paste0("H0_A: theta = ", format(H0_A, nsmall = 2)),
    paste0("H0_B: theta = ", format(H0_B, nsmall = 2)),
    paste0("H0_C: theta = ", format(H0_C, nsmall = 2))
  ),
  prop_rejeicao = c(
    mean(dec_A == "Rejeita H0"),
    mean(dec_B == "Rejeita H0"),
    mean(dec_C == "Rejeita H0")
  ),
  media_p  = c(mean(p_list_A), mean(p_list_B), mean(p_list_C)),
  mediana_p= c(median(p_list_A), median(p_list_B), median(p_list_C)),
  q05_p    = c(quantile(p_list_A, 0.05), quantile(p_list_B, 0.05), quantile(p_list_C, 0.05)),
  q95_p    = c(quantile(p_list_A, 0.95), quantile(p_list_B, 0.95), quantile(p_list_C, 0.95)),
  theta_hat_medio = c(mean(thA), mean(thB), mean(thC)),
  theta_real      = theta
)

# formatação
options(scipen = 999)
resumo_mc$prop_rejeicao   = format(round(resumo_mc$prop_rejeicao, 6), nsmall = 6)
resumo_mc$media_p         = format(round(resumo_mc$media_p,       6), nsmall = 6)
resumo_mc$mediana_p       = format(round(resumo_mc$mediana_p,     6), nsmall = 6)
resumo_mc$q05_p           = format(round(resumo_mc$q05_p,         6), nsmall = 6)
resumo_mc$q95_p           = format(round(resumo_mc$q95_p,         6), nsmall = 6)
resumo_mc$theta_hat_medio = format(round(resumo_mc$theta_hat_medio, 6), nsmall = 6)
resumo_mc$theta_real      = format(round(resumo_mc$theta_real,      6), nsmall = 6)

# Impressao da tabela
rownames(resumo_mc) = c(
  "Cenário A (H0_A exato)",
  "Cenário B (H0_B pouco desvio)",
  "Cenário C (H0_C muito desvio)"
)

noquote(resumo_mc)
\end{lstlisting}


    \begin{answer}[Questão 02]
    \end{answer}

    \begin{answer}[Questão 03]
    \end{answer}

     \begin{minipage}[c]{0.84\textwidth}
        \hspace{2em}%
        Os códigos desde estudo estão disponibilizados em \url{http://academic-codex.github.io/MAE5911-Estatistica-e-Machine-Learning}.
      \end{minipage}
      \hfill
      \begin{minipage}[c]{0.13\textwidth}
    \centering
        \includegraphics[width=0.7\linewidth]{figuras/qrcode.png}
      \end{minipage}
\end{document}