\documentclass[a4paper]{article}
\usepackage{student}
\usepackage{graphicx}
\usepackage{caption}
\usepackage[version=4]{mhchem}
\usepackage{tikz}
\usetikzlibrary{shapes.geometric, arrows.meta, positioning, decorations.pathreplacing}
\usepackage{enumitem}
\usepackage[utf8]{inputenc}
\usepackage{amsmath}
\usepackage{booktabs}
\usepackage{float}
\usepackage[portuguese]{babel}
\usepackage[T1]{fontenc}
\usepackage[utf8]{inputenc}
\usepackage[numbers,sort&compress]{natbib} % ou [authoryear]
\usepackage[alf]{abntex2cite} % citação ABNT autor-data
\usepackage{tcolorbox}
\tcbuselibrary{breakable}
\pagestyle{plain}

\tikzstyle{arrow} = [thick,->,>=stealth]

% Definindo o estilo de destaque com linhas pontilhadas
\tikzstyle{highlight} = [draw, dashed, thick, rectangle, rounded corners, inner sep=0.2cm, orange]


\tikzstyle{startstop} = [
    rectangle, rounded corners, minimum width=0.5cm,
    text centered, draw=black, fill=blue!10, font=\small
]
\tikzstyle{startstop_S} = [
    rectangle, rounded corners, minimum width=0.5cm, minimum height=0.8cm,
    text centered, draw=black, fill=green!30, font=\small
]
\tikzstyle{decision} = [
    diamond, aspect=2, draw=black, fill=orange!15, align=center,
    text centered, inner sep=0pt, font=\small
]
\tikzstyle{decision_S} = [
    diamond, aspect=2, draw=black, fill=orange!30, align=center,
    text centered, inner sep=0pt, font=\small
]
\tikzstyle{arrow} = [thick,->,>=stealth]



% Metadata
\date{\today}
\setmodule{MAE5911/IME: Fundamentos de Estatística e Machine Learning. \\ Prof.: Alexandre Galvão Patriota} 
\setterm{2o. semestre, 2025}

%-------------------------------%
% Other details
% TODO: Fill these
%-------------------------------%
\title{Lista 03 - 21/11}
\setmembername{Nara Avila Moraes}  % Fill group member names
\setmemberuid{5716734}  % Fill group member uids (same order)

%-------------------------------%
% Add / Delete commands and packages
% TODO: Add / Delete here as you need
%-------------------------------%
\usepackage{amsmath,amssymb,bm}

\newcommand{\KL}{\mathrm{KL}}
\newcommand{\R}{\mathbb{R}}
\newcommand{\E}{\mathbb{E}}
\newcommand{\T}{\top}

\newcommand{\expdist}[2]{%
        \normalfont{\textsc{Exp}}(#1, #2)%
    }
\newcommand{\expparam}{\bm \lambda}
\newcommand{\Expparam}{\bm \Lambda}
\newcommand{\natparam}{\bm \eta}
\newcommand{\Natparam}{\bm H}
\newcommand{\sufstat}{\bm u}

% Main document
\begin{document}
    % Add header
    \header{}

\textbf{Questão 01:} 

Considere uma amostra aleatória $(Y_1,X_1),\ldots,(Y_n,X_n)$ de $(Y,X)$ tal que a distribuição condicional
\[
Y \mid X = x \sim \mathcal{N}\big(\mu_{\theta}(x),\,\sigma^{2}_{\theta}(x)\big),
\]
e suponha que a distribuição de $X$ não contém informação sobre os parâmetros.

\begin{enumerate}
    \item Apresente uma \textbf{rede neural} que modele o \textbf{quantil de ordem $75\%$} (terceiro quartil) da distribuição condicional $Y \mid X = x$.  O código deve ser generalizável para qualquer quantil.

    \item Mostre a aplicação do método nos seguintes dados simulados em \textsf{R}:
    \begin{tcolorbox}[title={.r},breakable]
    \begin{verbatim}
    set.seed(32)

    n = 1000
    x = sort(runif(n, -4, 4))
    y = 3/(3+2*abs(x)^3)+exp(-x^2)+cos(x)*sin(x)+rnorm(n)*0.3
    \end{verbatim}
    \end{tcolorbox}
    \end{enumerate}
    \noindent
    Sugestão: reescreva a esperança e variância condicionais em termos do quantil
    e construa a função de verossimilhança apropriada.

        \begin{answer}[]

    \end{answer}


\end{document}
