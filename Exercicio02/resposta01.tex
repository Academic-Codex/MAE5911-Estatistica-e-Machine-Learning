\paragraph{(a) \(g(\theta)=P_\theta(Z=0)\)}

\[
  g(\theta)=P_\theta(Z=0)=1-\theta.
\]

Pelo teorema da \textbf{invariância do EMV}:
\[
  \widehat g_{MV}=g(\widehat\theta_{MV})=1-\widehat\theta_{MV}.
\]

Já encontramos que \(\widehat\theta_{MV}=\overline Z\), então:
\[
  \boxed{\widehat g_{MV}=1-\overline Z}.
\]



\paragraph{(b) \(g(\theta)=\mathrm{Var}_\theta(Z)\)}

\[\]
A variância é definida por:
\[
  \mathrm{Var}(Z)=E(Z^2)-E(Z)^2.
\]

Pela definição de esperança para variáveis discretas,
\[
  E(Z) = \sum_{z \in \{0,1\}} z \, P(Z = z).
\]

Mas $P(Z=1)=\theta$ e $P(Z=0)=1-\theta$, então:
\[
  E(Z)
  = 0 \cdot (1-\theta) + 1 \cdot \theta
  = \theta.
\]

\(E(Z^2)=E(Z)=\theta\) (pois \(Z\in\{0,1\}\)), então:
\[
  \mathrm{Var}_\theta(Z)=\theta-\theta^2=\theta(1-\theta).
\]

Portanto:
\[
  g(\theta)=\mathrm{Var}_\theta(Z)=\theta(1-\theta).
\]
Pela \textbf{invariância do EMV}:
\[
  \widehat g_{MV}=g(\widehat\theta_{MV})
  =\widehat\theta_{MV}(1-\widehat\theta_{MV}).
\]

Para \(\widehat\theta_{MV}=\overline Z\), encontramos:
\[
  \boxed{\widehat g_{MV}=\overline Z(1-\overline Z).}
\]



\paragraph{(c) Sejam os dados observados: \( (0, 0, 1, 0, 0, 1) \)}
\[\]
Para o cálculo da média: \( n = 6 \) e \( \sum z_i = 2 \), portanto
\[
  \hat\theta_{MV} = \bar{Z} = \frac{2}{6} = \frac{1}{3}.
\]
Assim:

\begin{itemize}
  \item[(a)] $\quad\hat g_{MV}=\widehat{P_{\theta}(Z=0)} = 1 - \bar{Z} = \frac{2}{3}$
  \item[(b)] $\quad\hat g_{MV}=\widehat{\mathrm{Var_{\theta}}}(Z) = \bar{Z}(1 - \bar{Z})
          = \frac{1}{3} \cdot \frac{2}{3} = \frac{2}{9}$
\end{itemize}


\paragraph{(d) Teste de hipótese \( H_0: \theta = 0.1 \)}
\[\]
Na ditribuição de Bernoulli, o parâmetro $\theta$ representa a probabilidade de sucesso, equivalente à média. Se supormos que a hipótese nula é verdadeira, o valor-p dos dados observados em relação a H0 é:

Dados amostrados:

\[
  (0, 0, 1, 0, 0, 1)
\]

Sob a hipótese nula \( \theta = 0.1 \), contruimos X:

\[
  X \sim \mathrm{Binomial}(n = 6,\, \theta = 0.1).
\]

A estatística de interesse é a média amostral ou probabilidade de sucessos. Em relação à amostra observada, é:

\[
  \bar{Z} = \frac{1}{n}\sum_{i=1}^{n} z_i =\frac{1}{6}\sum_{i=1}^{6} z_i = \frac{1}{6}*2=\frac{1}{3}.
\]


Vamos analisar a média da amostra contra a média testada (a média da hipótese nula). O valor-p então é definido a partir de:
no cenário em que $H0$ é válida, ou seja, temos uma distribuicao de Bernoulli com média = 0.1, qual a probabilidade de observar média $\bar{Z}=1/3$ ou maior?
O posicionamento da média observada em relação à média testada indica a direção do teste.


\[
  p = P_{H_0}\!\left( \bar{Z} \ge \frac{1}{3} \right).
\]

Como \quad$X = \sum_{i=1}^6 z_i$, \quad$\displaystyle \bar{Z} = \frac{1}{n}\sum_{i=1}^n z_i$ e\quad \(n=6\):

\[
  \bar{Z} = \frac{X}{6}.
\]

Então,

\[
  \bar{Z} \ge \frac{1}{3}
  \;\Longleftrightarrow\;
  \frac{X}{6} \ge \frac{1}{3}
  \;\Longleftrightarrow\;
  X \ge 2.
\]

Logo, o valor-\(p\) é:

\[
  p = P_{H_0}(X \ge 2),
\]

\[
  p = P(X=2) + P(X=3) + P(X=4) + P(X=5) + P(X=6).
\]

Cada termo é dado por:
\[
  P(X=k) = \binom{6}{k} (0.1)^k (0.9)^{6-k}.
\]

Portanto:
\[
  P(X = 2)
  = \binom{6}{2}(0.1)^2(0.9)^4
  = 15 \cdot 0.01 \cdot 0.6561
  = 0.098415.
\]

\[
  P(X = 3)
  = \binom{6}{3}(0.1)^3(0.9)^3
  = 20 \cdot 0.001 \cdot 0.729
  = 0.01458.
\]

\[
  P(X = 4)
  = \binom{6}{4}(0.1)^4(0.9)^2
  = 15 \cdot 0.0001 \cdot 0.81
  = 0.001215.
\]

\[
  P(X = 5)
  = \binom{6}{5}(0.1)^5(0.9)
  = 6 \cdot 0.00001 \cdot 0.9
  = 0.000054.
\]

\[
  P(X = 6)
  = \binom{6}{6}(0.1)^6
  = 1 \cdot 0.000001
  = 0.000001.
\]

Assim, o valor-p é:
\[
  p = 0.098415
  + 0.01458
  + 0.001215
  + 0.000054
  + 0.000001
  = 0.114265.
\]

Pela convervação da probabilidade, seria mais simples calcular o valor-p pelo seu valor complementar: $$\bar{p}=P_{H0}(X < 2)=(1 - p)$$
\[
  P(X=0) = (0.9)^6 = 0.531441, \qquad
  P(X=1) = \binom{6}{1}(0.1)(0.9)^5 = 0.354294.
\]

Logo:
\[
  \bar{p} = 1 - p = P(X=0) + P(X=1) = 0.531441 + 0.354294 = 0.885735.
\]

\[
  \boxed{\text{valor-p} = 1 - 0.885735 = 0.114265}
\]

---

Considerando um nível de significância $\alpha=0.05$ não há evidências suficientes para rejeitar H0, isto é, para supor que a média seja diferente de 0.1.

%%%%%%%%%%%%%%%%%%%%%%%%%%%%%%%%%%%%%%%%%%%%%%%%%%%%%%%%%%%%%%%%%%%%%%%%%%%%%%
