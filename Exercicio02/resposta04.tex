\paragraph{(a) $g(\theta)=\mathbb E_\theta(Z)$.}
\[\]

Pela definição de esperança para variáveis contínuas, sendo $h(z)$ a função de interesse:
\[
    E_\theta(h(Z)) = \int_{-\infty}^{\infty} h(z)\, f(z;\theta)\,dz.
\]

Portanto, a esperança do primeiro momento de $Z$, para a distribuição Beta$(\theta,1)$, é:
\[
    E_\theta(Z)
    = \int_0^1 z \,\theta z^{\theta-1} dz
    = \theta \int_0^1 z^{\theta} dz
    = \theta \left[\frac{z^{\theta+1}}{\theta+1}\right]_0^1
    = \frac{\theta}{\theta+1},
\]


Então:

\[
    g(\theta)=\mathbb{E}_\theta(Z)=\frac{\theta}{\theta+1}.
\]

Pela \textbf{invariância do EMV}, o estimador de máxima verossimilhança de $g(\theta)$ é:

\[
    \widehat g_{MV} = g(\widehat\theta_{MV})
    = \frac{\widehat\theta_{MV}}{\widehat\theta_{MV}+1}
    = \frac{-\frac{n}{\sum \log z_i}}{1 - \frac{n}{\sum \log z_i}}
    = \frac{n}{n - \sum_{i=1}^n \log z_i}.
\]

\[
    \boxed{\displaystyle
        \widehat g_{MV}
        = \frac{n}{\,n - \sum_{i=1}^n \log z_i\,}.
    }
\]

\paragraph{(b) $g(\theta)=P_\theta(Z>0{,}3)$}
\[\]

Integrando a função densidade de probabilidade $f(z,\theta)$ para encontrar
$P_\theta(Z>0{,}3)$, temos:
\[
    g(\theta)
    = P_\theta(Z>0{,}3)
    = \int_{0{,}3}^{1} f(z,\theta)\,dz
    = \int_{0{,}3}^{1} \theta z^{\theta-1}\,dz.
\]

Calculando a integral:
\[
    \int_{0{,}3}^{1} \theta z^{\theta-1}\,dz
    = \theta\left[\frac{z^{\theta}}{\theta}\right]_{0{,}3}^{1}
    = \bigl[z^{\theta}\bigr]_{0{,}3}^{1}
    = 1 - (0{,}3)^{\theta}.
\]

Logo,
\[
    g(\theta) = 1 - (0{,}3)^{\theta}.
\]

Pela \textbf{invariância do EMV}, o estimador de máxima verossimilhança de \(g(\theta)\) é:
\[
    \widehat g_{MV} = g(\widehat\theta_{MV})
    = 1 - (0{,}3)^{\widehat\theta_{MV}} = 1 - (0{,}3)^{-\frac{n}{\sum_{i=1}^n \log z_i}}
\]

\[
    \boxed{
        \widehat g_{MV} = 1 - (0{,}3)^{-\frac{n}{\sum_{i=1}^n \log z_i}}.}
\]


%%%%%%%%%%

\paragraph{(c) $g(\theta)=P_\theta(0<Z<0{,}1)$}
\[\]

Integrando a função densidade de probabilidade $f(z,\theta)$ para encontrar
$P_\theta(0<Z<0{,}1)$, temos:
\[
    g(\theta)
    = P_\theta(0<Z<0{,}1)
    = \int_{0}^{0{,}1} f(z,\theta)\,dz
    = \int_{0}^{0{,}1} \theta z^{\theta-1}\,dz.
\]

Calculando a integral:
\[
    \int_{0}^{0{,}1} \theta z^{\theta-1}\,dz
    = \theta\left[\frac{z^{\theta}}{\theta}\right]_{0}^{0{,}1}
    = \bigl[z^{\theta}\bigr]_{0}^{0{,}1}
    = (0{,}1)^{\theta} - 0
    = (0{,}1)^{\theta}.
\]

Logo,
\[
    g(\theta) = (0{,}1)^{\theta}.
\]

Pela \textbf{invariância do EMV}, o estimador de máxima verossimilhança de \(g(\theta)\) é:
\[
    \widehat g_{MV} = g(\widehat\theta_{MV})
    = (0{,}1)^{\widehat\theta_{MV}}
    = (0{,}1)^{-\frac{n}{\sum_{i=1}^n \log z_i}}.
\]

\[
    \boxed{
        \widehat g_{MV}
        = (0{,}1)^{-\frac{n}{\sum_{i=1}^n \log z_i}}
    }.
\]

\paragraph{(d) $g(\theta)=\mathrm{Var}_\theta(Z)$.}
\[\]
A variância é definida por:

\[
    \mathrm{Var}(Z) = E_\theta(Z^2) - \bigl(E_\theta(Z)\bigr)^2.
\]

Pela definição de esperança para variáveis contínuas, sendo $h(z)$ a função de interesse:
\[
    E_\theta(h(Z)) = \int_{-\infty}^{\infty} h(z)\, f(z;\theta)\,dz.
\]

Em particular, a esperança do primeiro e segundo momentos de $Z$, para a Beta$(\theta,1)$, são:
\[
    E_\theta(Z)
    = \frac{\theta}{\theta+1},
\]
e
\[
    E_\theta(Z^2)
    = \int_0^1 z^{2}\,\theta z^{\theta-1} dz
    = \theta \int_0^1 z^{\theta+1} dz
    = \theta \left[\frac{z^{\theta+2}}{\theta+2}\right]_0^1
    = \frac{\theta}{\theta+2}.
\]


Então:
\[
    \mathrm{Var}\theta(Z)
    = E_\theta(Z^2) - \bigl(E_\theta(Z)\bigr)^2
    = \frac{\theta}{\theta+2}
    - \left(\frac{\theta}{\theta+1}\right)^2
    = \frac{\theta}{(\theta+1)^2(\theta+2)}.
\]

Portanto,
\[
    g(\theta)=\frac{\theta}{(\theta+1)^2(\theta+2)}.
\]

Pela \textbf{invariância do EMV}, o estimador de máxima verossimilhança de \(g(\theta)\) é:
\[
    \widehat g_{MV}(\theta) = g(\hat\theta_{MV}) = \frac{\hat\theta_{MV}}{(\hat\theta_{MV}+1)^2(\hat\theta_{MV}+2)} .
\]

\[
    \boxed{
        \widehat g_{MV}
        = \frac{\hat\theta_{MV}}{(\hat\theta_{MV}+1)^2(\hat\theta_{MV}+2).
        }}
\]

%---------------------------------------------------
\paragraph{(e) Sejam os dados observados: $(0.12,\,0.50,\,0.20,\,0.23,\,0.30,\,0.11)$}
\[\]
Com $n=6$,
\[
    \sum_{i=1}^n\log z_i
    =\log(0.12)+\log(0.50)+\log(0.20)+\log(0.23)+\log(0.30)+\log(0.11)
    \approx -9.3038.
\]
Logo,
\[
    \widehat\theta_{MV}
    =-\frac{6}{-9.3038}\approx 0.6449.
\]

Assim:

\begin{itemize}
    \item[(a)] $\displaystyle \widehat g_{MV}=\frac{\widehat\theta_{MV}}{1+\widehat\theta_{MV}}
              \approx \frac{0.6449}{1.6449}\approx 0.3921$

    \item[(b)] $\displaystyle \widehat g_{MV}=1-(0.3)^{\widehat\theta_{MV}}
              \approx 1-(0.3)^{0.6449}\approx 0.5400$

    \item[(c)] $\displaystyle \widehat g_{MV}=(0.1)^{\widehat\theta_{MV}}
              \approx (0.1)^{0.6449}\approx 0.2265$

    \item[(d)] $\displaystyle \widehat g_{MV}=\frac{\widehat\theta_{MV}}{(\widehat\theta_{MV}+1)^2(\widehat\theta_{MV}+2)}
              \approx \frac{0.6449}{(1.6449)^2(2.6449)}\approx 0.0901$
\end{itemize}
