\textbf{Questão 04.} Seja $(Z_1,\ldots,Z_n)$ uma amostra i.i.d. de densidade
\[
f_\theta(x)=\theta\,x^{\theta-1},\qquad x\in(0,1),\ \ \theta>0,
\]
e $f_\theta(x)=0$ caso contrário. (Trata-se de uma Beta$(\theta,1)$.)

%---------------------------------------------------
\paragraph{EMV de $\theta$.}
A verossimilhança é
\[
L(\theta;\mathbf z)=\prod_{i=1}^n \theta\,z_i^{\theta-1}
=\theta^n\prod_{i=1}^n z_i^{\theta-1}
=\theta^n\Big(\prod_{i=1}^n z_i\Big)^{\theta-1}.
\]
A log-verossimilhança é
\[
\ell(\theta)=\log L(\theta;\mathbf z)
=n\log\theta+(\theta-1)\sum_{i=1}^n\log z_i .
\]
Derivando e igualando a zero,
\[
\ell'(\theta)=\frac{n}{\theta}+\sum_{i=1}^n\log z_i=0
\quad\Longrightarrow\quad
\widehat\theta_{MV}=-\,\frac{n}{\sum_{i=1}^n\log Z_i}.
\]
A segunda derivada é
\[
\ell''(\theta)=-\,\frac{n}{\theta^2}<0\quad(\theta>0),
\]
logo o ponto crítico é máximo global. Assim,
\[
\boxed{\ \widehat\theta_{MV}=-\,\dfrac{n}{\sum_{i=1}^n\log Z_i}\ }.
\]

%---------------------------------------------------
\paragraph{Momentos e probabilidades.}
Para $X\sim f_\theta$, para $a>- \theta$,
\[
\mathbb E_\theta(X^a)=\int_0^1 x^a\,\theta x^{\theta-1}dx
=\theta\int_0^1 x^{a+\theta-1}dx
=\frac{\theta}{a+\theta}.
\]
Em particular,
\[
\mathbb E_\theta(X)=\frac{\theta}{\theta+1},\qquad
\mathbb E_\theta(X^2)=\frac{\theta}{\theta+2},\qquad
\mathrm{Var}_\theta(X)=\frac{\theta}{(\theta+1)^2(\theta+2)}.
\]
Para $0<a<b\le1$,
\[
P_\theta(a<X<b)=\int_a^b \theta x^{\theta-1}dx
=\Big[x^\theta\Big]_a^b=b^\theta-a^\theta.
\]

%---------------------------------------------------
\paragraph{(a) $g(\theta)=\mathbb E_\theta(Z)$.}
\[
\mathbb E_\theta(Z)=\frac{\theta}{\theta+1}
\quad\Longrightarrow\quad
\boxed{\ \widehat g_{(a)}=\frac{\widehat\theta_{MV}}{\widehat\theta_{MV}+1}\ }.
\]

\paragraph{(b) $g(\theta)=P_\theta(Z>0.3)$.}
\[
P_\theta(Z>0.3)=1-P_\theta(0<Z\le0.3)=1-(0.3)^\theta
\quad\Longrightarrow\quad
\boxed{\ \widehat g_{(b)}=1-(0.3)^{\widehat\theta_{MV}}\ }.
\]

\paragraph{(c) $g(\theta)=P_\theta(0<Z<0.1)$.}
\[
P_\theta(0<Z<0.1)=(0.1)^\theta
\quad\Longrightarrow\quad
\boxed{\ \widehat g_{(c)}=(0.1)^{\widehat\theta_{MV}}\ }.
\]

\paragraph{(d) $g(\theta)=\mathrm{Var}_\theta(Z)$.}
\[
\mathrm{Var}_\theta(Z)=\frac{\theta}{(\theta+1)^2(\theta+2)}
\quad\Longrightarrow\quad
\boxed{\ \widehat g_{(d)}=\frac{\widehat\theta_{MV}}
{(\widehat\theta_{MV}+1)^2(\widehat\theta_{MV}+2)}\ }.
\]

%---------------------------------------------------
\paragraph{(e) Estimativas numéricas.}
Dados: $(0.12,\,0.50,\,0.20,\,0.23,\,0.30,\,0.11)$.
Com $n=6$,
\[
\sum_{i=1}^n\log z_i
=\log(0.12)+\log(0.50)+\log(0.20)+\log(0.23)+\log(0.30)+\log(0.11)
\approx -9.3038.
\]
Logo,
\[
\widehat\theta_{MV}
=-\frac{6}{-9.3038}\approx 0.6449.
\]
Então,
\[
\widehat g_{(a)}=\frac{\widehat\theta}{1+\widehat\theta}
\approx \frac{0.6449}{1.6449}\approx 0.3921,
\]
\[
\widehat g_{(b)}=1-(0.3)^{\widehat\theta}
\approx 1-(0.3)^{0.6449}\approx 0.5400,
\]
\[
\widehat g_{(c)}=(0.1)^{\widehat\theta}
\approx (0.1)^{0.6449}\approx 0.2265,
\]
\[
\widehat g_{(d)}=\frac{\widehat\theta}{(\widehat\theta+1)^2(\widehat\theta+2)}
\approx \frac{0.6449}{(1.6449)^2(2.6449)}\approx 0.0901.
\]

\medskip
\noindent
\textit{Resumo:}\;
$\widehat\theta_{MV}=-n/\sum\log Z_i\approx 0.6449$ e,
por invariância, as estimativas de (a)--(d) são os valores de $g(\theta)$
avaliados em $\widehat\theta$ como mostrado acima.