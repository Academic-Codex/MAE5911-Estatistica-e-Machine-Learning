\documentclass[a4paper]{article}
\usepackage{student}
\usepackage{graphicx}
\usepackage{caption}
\usepackage[version=4]{mhchem}
\usepackage{tikz}
\usetikzlibrary{shapes.geometric, arrows.meta, positioning, decorations.pathreplacing}
\usepackage{enumitem}
\usepackage[utf8]{inputenc}
\usepackage{amsmath}
\usepackage{booktabs}
\usepackage{float}
\usepackage[portuguese]{babel}
\usepackage[T1]{fontenc}
\usepackage[utf8]{inputenc}
\usepackage[numbers,sort&compress]{natbib} % ou [authoryear]
\usepackage[alf]{abntex2cite} % citação ABNT autor-data
\pagestyle{plain}

\tikzstyle{arrow} = [thick,->,>=stealth]

% Definindo o estilo de destaque com linhas pontilhadas
\tikzstyle{highlight} = [draw, dashed, thick, rectangle, rounded corners, inner sep=0.2cm, orange]


\tikzstyle{startstop} = [
    rectangle, rounded corners, minimum width=0.5cm,
    text centered, draw=black, fill=blue!10, font=\small
]
\tikzstyle{startstop_S} = [
    rectangle, rounded corners, minimum width=0.5cm, minimum height=0.8cm,
    text centered, draw=black, fill=green!30, font=\small
]
\tikzstyle{decision} = [
    diamond, aspect=2, draw=black, fill=orange!15, align=center,
    text centered, inner sep=0pt, font=\small
]
\tikzstyle{decision_S} = [
    diamond, aspect=2, draw=black, fill=orange!30, align=center,
    text centered, inner sep=0pt, font=\small
]
\tikzstyle{arrow} = [thick,->,>=stealth]



% Metadata
\date{\today}
\setmodule{MAE5911/IME: Fundamentos de Estatística e Machine Learning. \\ Prof.: Alexandre Galvão Patriota} 
\setterm{2o. semestre, 2025}

%-------------------------------%
% Other details
% TODO: Fill these
%-------------------------------%
\title{Exercício 02 - 15/10}
\setmembername{Nara Avila Moraes}  % Fill group member names
\setmemberuid{5716734}  % Fill group member uids (same order)

%-------------------------------%
% Add / Delete commands and packages
% TODO: Add / Delete here as you need
%-------------------------------%
\usepackage{amsmath,amssymb,bm}

\newcommand{\KL}{\mathrm{KL}}
\newcommand{\R}{\mathbb{R}}
\newcommand{\E}{\mathbb{E}}
\newcommand{\T}{\top}

\newcommand{\expdist}[2]{%
        \normalfont{\textsc{Exp}}(#1, #2)%
    }
\newcommand{\expparam}{\bm \lambda}
\newcommand{\Expparam}{\bm \Lambda}
\newcommand{\natparam}{\bm \eta}
\newcommand{\Natparam}{\bm H}
\newcommand{\sufstat}{\bm u}

% Main document
\begin{document}
    % Add header
    \header{}

\textbf{Questão 01:}  
Seja $(Z_1, \ldots, Z_n)$ uma amostra aleatória de $Z \sim \text{Ber}(\theta)$, $\theta \in (0,1)$.
\begin{itemize}
  \item[(a)] Encontre o EMV para $g(\theta) = P_\theta(Z = 0)$.
  \item[(b)] Encontre o EMV para $g(\theta) = \mathrm{Var}_\theta(Z)$.
  \item[(c)] Considere que os dados foram observados (0, 0, 1, 0, 0, 1). Encontre as estimativas de MV nos itens acima.
  \item[(d)] Construa o valor-p para a hipótese H: ``$\theta = 0.1$'' usando os dados do item anterior.
\end{itemize}
    \begin{answer}[]
 O estimador via função de máxima verossimilhança foi proposto por Fisher, que demonstrou a superioridade do deste método referente a outros métodos estimadores, como o método de momentos. 
Este estimador é utilizado sobretudo quando a informação da distribuição de probabilidade do modelo é conhecida. 
Ela tem propriedades como eficiência assintótica, consistência, distribuição normal e invariancia [Notas de aula pg. 60]. 
O estimador de máxima verossimilhança representa uma estimativa para o valor do parâmetro $\theta$ que torna os dados observados o mais plausível possível. 
\\[0.5em]
Devido a propriedade de invariancia do EMV, basta encontrarmos EMV para a letra (a) e depois calcular a letra (b) utilizando este estimador como parâmetro.


Seja \( (Z_1, \dots, Z_n) \) a amostra aleatória de 
\[
Z_i \sim \mathrm{Bernoulli}(\theta), \quad \theta \in (0,1).
\]

A função de verossimilhança, que é a função de densidade de probabilidade conjunta do modelo, é dada por
\[
L(\theta; z_1, \dots, z_n)
= \prod_{i=1}^n \theta^{z_i}(1-\theta)^{1-z_i}.
\]

Para encontrar o estimador de máxima verossimilhança, precisamos maximizar \( L(\theta) \) em relação a \( \theta \). Mas como esta função se trata de um produtório,
é mais fácil maximizar a função log-verossimilhança, que por ser monotonicamente crescente, preserva o valor de theta no ponto máximo \( L(\theta) \).


\text{Tomando o logaritmo natural em ambos os lados:}

\[
\ell(\theta)
= \log L(\theta; z_1, \ldots, z_n)
= \log\!\left(\prod_{i=1}^{n} \theta^{z_i}(1-\theta)^{1-z_i}\right).
\]


\text{Sabendo que } 
\(
\log(ab) = \log a + \log b \text{ e que } 
\log\!\left(\prod_i a_i\right) = \sum_i \log a_i, \text{ temos:}
\)

\[
\ell(\theta)
= \sum_{i=1}^{n} \log\!\big(\theta^{z_i}(1-\theta)^{1-z_i}\big).
\]

\text{Aplicando novamente a propriedade do logaritmo de um produto: }

\[
\ell(\theta)
= \sum_{i=1}^{n} \big[\log(\theta^{z_i}) + \log((1-\theta)^{1-z_i})\big].
\]

\(
\text{Sabendo que } \log(a^b) = b\log a, \text{ obtemos:}
\)

\[
\ell(\theta)
= \sum_{i=1}^{n} \big[z_i \log \theta + (1 - z_i)\log(1-\theta)\big].
\]

\text{Por fim, separando as somas:}

\[
\ell(\theta)
= \left(\sum_{i=1}^{n} z_i\right)\log \theta
+ \left(\sum_{i=1}^{n}(1-z_i)\right)\log(1-\theta).
\]

A log-verossimilhança correspondente é
\[
\boxed{\ell(\theta) 
= \sum_{i=1}^n \big[z_i \log\theta + (1-z_i)\log(1-\theta)\big].}
\]

Para derivar em relação a \(\theta\), sabemos que 
\(\dfrac{d}{d\theta}\log\theta=\dfrac1\theta\) e 
\(\dfrac{d}{d\theta}\log(1-\theta)= -\dfrac1{1-\theta}\), obtemos,
termo a termo:

\[
\frac{d}{d\theta}\Big[z_i\log\theta\Big]
= z_i\,\frac1\theta
\qquad\text{e}\qquad
\frac{d}{d\theta}\Big[(1-z_i)\log(1-\theta)\Big]
= (1-z_i)\Big(-\frac1{1-\theta}\Big)
= -\,\frac{1-z_i}{1-\theta}.
\]
Logo,
\[
\ell'(\theta)
=\sum_{i=1}^n\left(\frac{z_i}{\theta}-\frac{1-z_i}{1-\theta}\right)
= \frac{\sum_{i=1}^n z_i}{\theta} - \frac{\sum_{i=1}^n(1-z_i)}{1-\theta}
= \frac{\sum_i z_i}{\theta} - \frac{n-\sum_i z_i}{1-\theta}.
\]

\text{Igualando a zero para encontrar o} \textbf{ponto crítico.}
\[
\ell'(\theta)=0
\;\Longleftrightarrow\;
\frac{\sum_i z_i}{\theta}
= \frac{n-\sum_i z_i}{1-\theta}
\;\Longleftrightarrow\;
(1-\theta)\sum_i z_i=\theta\,(n-\sum_i z_i).
\]
Expandindo e rearranjando:
\[
\sum_i z_i - \theta\sum_i z_i = \theta n - \theta\sum_i z_i
\;\Longleftrightarrow\;
\sum_i z_i = \theta n
\;\Longleftrightarrow\;
\widehat\theta_{MV}=\frac1n\sum_{i=1}^n z_i=\overline Z.
\]


Obtivemos o estimador de máxima verossimilhança para a probabilidade de sucesso \(\theta\):
\[
\boxed{\hat{\theta}_{MV} = \bar{Z} = \frac{1}{n}\sum_{i=1}^n Z_i.}
\]

O teste de hipótese, mede o quão improvável seria observar $\hat{\theta}_{MV}$ se a hipótese nula fosse verdadeira. O valor-p é esta medida de improbabilidade.
\\[1em]
Em condições de regularidade, o valor-$p$ segue uma distribuição uniforme em $(0,1)$.
Isso significa que, se o seu valor é muito pequeno, por exemplo $p \leq 0{,}01$,
então o resultado observado é um evento raro sob a hipótese nula $H_0$,
ou, alternativamente, que a hipótese nula está incorreta.
Nessa situação, existem evidências para rejeitar $H_0$.


---

(a) \( g(\theta) = P_\theta(Z=0) \)

Pela \textbf{propriedade de invariância do EMV}, temos:
\[
\boxed{\hat{g}_{MV} = g(\hat{\theta}_{MV}) = 1 - \bar{Z}.}
\]

---

(b) \( g(\theta) = \mathrm{Var}_\theta(Z) \)

Sabemos que \( \mathrm{Var}_\theta(Z) = \theta(1 - \theta) \).  
Logo, aplicando novamente a invariância:
\[
\boxed{\hat{g}_{MV} = g(\hat{\theta}_{MV}) = \bar{Z}(1 - \bar{Z}).}
\]

---

(c) Sejam os dados observados: \( (0, 0, 1, 0, 0, 1) \)

Temos \( n = 6 \) e \( \sum z_i = 2 \), portanto
\[
\hat{\theta}_{MV} = \bar{Z} = \frac{2}{6} = \frac{1}{3} \approx 0.3333.
\]
Assim:
\[
\widehat{P(Z=0)} = 1 - \hat{\theta}_{MV} = \frac{2}{3} \approx 0.6667,
\]
\[
\widehat{\mathrm{Var}}(Z) = \hat{\theta}_{MV}(1 - \hat{\theta}_{MV})
= \frac{1}{3} \cdot \frac{2}{3} = \frac{2}{9} \approx 0.2222.
\]

---

(d) Teste de hipótese \( H_0: \theta = 0.1 \)

Queremos verificar se temos evidência a partir dos dados observados para rejeitar a hipótese nula 
de que a probabilidade de sucesso é $\theta = 0.1$.

Sabemos que, sob $H_0$, a estatística
\[
X = \sum_{i=1}^{n} Z_i
\]
segue uma distribuição $\text{Binomial}(n, \theta)$.

Com $n = 6$ e $x_{\text{obs}} = 2$ sucessos observados, temos:
\[
X \sim \text{Binomial}(6, 0.1).
\]

---
 
O valor-p é a probabilidade de observar um valor da estatística $T(Z_n)$ 
\textit{tão ou mais extremo quanto o observado}, assumindo que $H_0$ é verdadeira:
\[
\boxed{
\text{valor-}p(H_0, \mathbf{z}_n)
= \sup_{\theta \in \Theta_0}
P^{(n)}_{\theta}\!\big(
T_{H_0}(\mathbf{Z}_n)
\ge
T_{H_0}(\mathbf{z}_n)
\big).
}
\]
No contexto binomial, a estatística de teste é o número de sucessos $X$. Vamos avaliar a probabilidade de eventos
com parâmetro de sucesso a partir da amostra observada até eventos mais extremos uma vez que a amostra observada já está acima do esperado, em relação à hipotese nula. Portanto vamos avaliar:
\[
p = P_{H_0}(X \ge 2).
\]

---
\\
Calculando o valor-p pelo seu valor complementar:
\[
P(X=0) = (0.9)^6 = 0.531441, \qquad 
P(X=1) = \binom{6}{1}(0.1)(0.9)^5 = 0.354294.
\]

Logo, a probabilidade testada é o complemento da soma dessas probabilidades:
\[
valor-p = 1 - [P(X=0) + P(X=1)] = 1 - (0.531441 + 0.354294) = 0.114265.
\]

\[
\boxed{\text{valor-p} = 0.114265}
\]

---

Este resultado \textbf{não é improvável} sob $H_0$ (<0.05).  
Portanto, não há evidência para rejeitar  $H_0$, isto é, não evidências de que $\theta = 0.1$ seja incompatível com os dados.

\[
\boxed{\text{Conclusão: Não há evidências suficientes para rejeitar } H_0 : \theta = 0.1.}
\]

    \end{answer}

\textbf{Questão 02:}  
Seja $(Z_1, \ldots, Z_n)$ uma amostra aleatória de $Z \sim \text{Exp}(\theta)$, $\theta \in (0,\infty)$.
\begin{itemize}
  \item[(a)] Encontre o EMV para $g(\theta) = P_\theta(Z > 1)$.
  \item[(b)] Encontre o EMV para $g(\theta) = P_\theta(0.1 < Z < 1)$.
  \item[(c)] Encontre o EMV para $g(\theta) = \mathrm{Var}_\theta(Z)$.
  \item[(d)] Considere que os dados foram observados (0.2, 0.6, 0.3, 0.2, 0.8, 0.12). Encontre um IC aproximado de 95\% de confiança para $g(\theta)$ nos itens acima.
  \item[(e)] Faça uma simulação de Monte Carlo para verificar se os IC's aproximados obtidos no passo anterior têm cobertura próxima do nível de confiança estabelecido. Caso não tenham, proponha um tamanho amostral que produza IC's mais confiáveis para cada caso.
\end{itemize}

    \begin{answer}[]
Seja $Z_1,\ldots,Z_n \stackrel{\text{iid}}{\sim}\mathrm{Exp}(\theta)$, com $\theta>0$.
A função densidade de probabilidade é
\[
f(z\mid\theta)=\theta\,e^{-\theta z}\,\mathbf 1_{\{z\ge 0\}},
\]
onde $\mathbf 1_{\{z\ge 0\}}$ é a função indicadora (vale 1 se $z\ge 0$ e 0 caso contrário).

Como as observações são independentes e identicamente distribuídas,
a \textbf{função de verossimilhança}, que é a função de densidade de probabilidade conjunta é o produto das densidades individuais:
\[
L(\theta;z_1,\ldots,z_n)
=\prod_{i=1}^n f(z_i\mid\theta)
=\prod_{i=1}^n \theta\,e^{-\theta z_i}\,\mathbf 1_{\{z_i\ge 0\}}.
\]

Aplicando a propriedade distributiva do produto, separamos os fatores:
\[
L(\theta;z)
=\Big(\prod_{i=1}^n \mathbf 1_{\{z_i\ge 0\}}\Big)
\Big(\prod_{i=1}^n \theta e^{-\theta z_i}\Big).
\]

Como o termo indicador $\prod \mathbf 1_{\{z_i\ge 0\}}$ não depende de $\theta$,
ele não influencia a maximização e pode ser ignorado. Assim:
\[
L(\theta;z)
=\prod_{i=1}^n (\theta e^{-\theta z_i}).
\]

Usando a propriedade $\prod(ab)=\prod a \cdot \prod b$, obtemos:
\[
L(\theta;z)
=\Big(\prod_{i=1}^n \theta\Big)
\Big(\prod_{i=1}^n e^{-\theta z_i}\Big).
\]

Aplicando:
1. $\prod_{i=1}^n \theta = \theta^n$ (produto de n fatores iguais),  
2. $\prod e^{a_i} = e^{\sum a_i}$ (exponencial do somatório),  

segue que:
\[
\boxed{
L(\theta;z)
=\theta^n \exp\!\left(-\theta \sum_{i=1}^n z_i\right).}
\]

---

Para encontrar a função \textbf{Log-verossimilhança}, aplicamos $\log$ em ambos os lados, e usamos as propriedades do logaritmo:

1. $\log(ab)=\log a+\log b$  
2. $\log(a^b)=b\log a$  
3. $\log(e^x)=x$  
4. $\log(\prod a_i)=\sum \log a_i$

\[
\ell(\theta)=\log L(\theta;z)
=\log(\theta^n)+\log\!\left(\exp\!\left[-\theta\sum_{i=1}^n z_i\right]\right).
\]

Aplicando $\log(a^b)=b\log a$ no primeiro termo
e $\log(e^x)=x$ no segundo termo, obtemos:
\[
\ell(\theta)=n\log\theta-\theta\sum_{i=1}^n z_i.
\]

Para maximizar, derivamos em relação a $\theta$, aplicando as regras de derivação:
\[
\frac{d}{d\theta}\log\theta=\frac{1}{\theta}, \qquad
\frac{d}{d\theta}(a\theta)=a.
\]

Logo:
\[
\boxed{
\ell'(\theta)
=\frac{d}{d\theta}[n\log\theta-\theta\sum_{i=1}^n z_i]
=n\frac{1}{\theta}-\sum_{i=1}^n z_i
=\frac{n}{\theta}-\sum_{i=1}^n z_i.}
\]

Igualamos a zero para encontrar o \textbf{ponto crítico}. 
\[
\ell'(\theta)=0
\;\Longleftrightarrow\;
\frac{n}{\theta}-\sum_{i=1}^n z_i=0
\;\Longleftrightarrow\;
\frac{n}{\theta}=\sum_{i=1}^n z_i
\;\Longleftrightarrow\;
\widehat\theta_{MV}=\frac{n}{\sum_{i=1}^n z_i}
=\frac{1}{\overline Z}.
\]

---

\paragraph{Derivada segunda (verificando máximo).}
Derivando novamente:
\[
\ell''(\theta)
=\frac{d}{d\theta}\!\left(\frac{n}{\theta}-\sum_{i=1}^n z_i\right)
=-\frac{n}{\theta^2}.
\]

Como $\ell''(\theta)<0$ para $\theta>0$,
o ponto crítico corresponde a um máximo.

---

O estimador de máxima verossimilhança é:
\[
\boxed{\widehat\theta_{MV}=\frac{n}{\sum_{i=1}^n z_i}
=\frac{1}{\overline Z}},
\]

\medskip
\textbf{Invariância do EMV.} Para $g(\theta)$, o EMV é $\widehat g=g(\widehat\theta_{MV})$.

\bigskip
\textbf{(a) } $g(\theta)=P_\theta(Z>1)$.
Como $P_\theta(Z>1)=e^{-\theta}$, então
\[
\widehat g_{(a)}=e^{-\widehat\theta_{MV}}.
\]

\medskip
\textbf{(b) } $g(\theta)=P_\theta(0.1<Z<1)$.
Como $P_\theta(a<Z<b)=e^{-\theta a}-e^{-\theta b}$,
\[
\widehat g_{(b)}=e^{-0.1\,\widehat\theta_{MV}}-e^{-\widehat\theta_{MV}}.
\]

\medskip
\textbf{(c) } $g(\theta)=\mathrm{Var}_\theta(Z)$.
Para Exponencial($\theta$) com taxa, $\mathrm{Var}(Z)=1/\theta^2$; logo
\[
\widehat g_{(c)}=\frac{1}{\widehat\theta_{MV}^2}=\overline Z^{\,2}.
\]

\bigskip
\textbf{(d) ICs aproximados de 95\% para $g(\theta)$ (usando o método delta).}
Para a Exponencial, a informação de Fisher é $I(\theta)=\tfrac{n}{\theta^2}$,
de modo que
\[
\widehat\theta_{MV}\ \dot\sim\ N\!\left(\theta,\ \frac{\theta^2}{n}\right).
\]
Pelo método delta, para $g$ diferenciável:
\[
\widehat g\ \dot\sim\ N\!\left(g(\theta),\ \frac{\theta^2}{n}\,[g'(\theta)]^2\right),
\quad\text{e usamos } \theta\leftarrow\widehat\theta_{MV}.
\]
Assim:

\smallskip
\begin{itemize}
\item[(a)] $g(\theta)=e^{-\theta}$, $g'(\theta)=-e^{-\theta}$. Então
\[
\widehat{\mathrm{Var}}(\widehat g_{(a)})
=\frac{\widehat\theta_{MV}^2}{n}\,e^{-2\widehat\theta_{MV}},
\qquad
\text{IC}_{95\%}:\ \widehat g_{(a)}\ \pm\ 1.96\,
\sqrt{\frac{\widehat\theta_{MV}^2}{n}\,e^{-2\widehat\theta_{MV}}}.
\]
\item[(b)] $g(\theta)=e^{-0.1\theta}-e^{-\theta}$,
\(
g'(\theta)=-0.1\,e^{-0.1\theta}+e^{-\theta}.
\)
Então
\[
\widehat{\mathrm{Var}}(\widehat g_{(b)})
=\frac{\widehat\theta_{MV}^2}{n}\,\bigl[-0.1\,e^{-0.1\widehat\theta_{MV}}+e^{-\widehat\theta_{MV}}\bigr]^2,
\]
\[
\text{IC}_{95\%}:\ \widehat g_{(b)}\ \pm\ 1.96\,
\sqrt{\frac{\widehat\theta_{MV}^2}{n}\,\bigl[-0.1\,e^{-0.1\widehat\theta_{MV}}+e^{-\widehat\theta_{MV}}\bigr]^2}.
\]
\item[(c)] $g(\theta)=1/\theta^2$, $g'(\theta)=-2/\theta^3$. Então
\[
\widehat{\mathrm{Var}}(\widehat g_{(c)})
=\frac{4}{n\,\widehat\theta_{MV}^{\,4}},
\qquad
\text{IC}_{95\%}:\ \widehat g_{(c)}\ \pm\ 1.96\,\sqrt{\frac{4}{n\,\widehat\theta_{MV}^{\,4}}}.
\]
\end{itemize}

\medskip
\textbf{Aplicando aos dados} $(0.2,0.6,0.3,0.2,0.8,0.12)$:
\[
n=6,\quad \textstyle\sum z_i=2.22,\quad \overline Z=0.37,\quad
\widehat\theta_{MV}=\frac{n}{\sum z_i}=\frac{6}{2.22}\approx 2.7027.
\]
\emph{Estimativas plug-in:}
\[
\widehat g_{(a)}=e^{-\widehat\theta_{MV}}\approx 0.0668,\qquad
\widehat g_{(b)}=e^{-0.1\widehat\theta_{MV}}-e^{-\widehat\theta_{MV}}\approx 0.6963,\qquad
\widehat g_{(c)}=\overline Z^{\,2}=0.1369.
\]
\emph{ICs (delta, 95\%):}
\[
\text{(a)}\ \ [0,\ 0.211]\ \ (\text{truncado a }[0,1]),
\qquad
\text{(b)}\ \ [0.676,\ 0.717],
\qquad
\text{(c)}\ \ [0,\ 0.356]\ \ (\text{truncado a }[0,\infty)).
\]
\textit{Obs.:} Com $n=6$ o delta pode produzir limites fora do espaço paramétrico; é
padrão truncar aos limites naturais.

\bigskip
\textbf{(e) Cobertura por Monte Carlo (plano de simulação).}
Para verificar a cobertura empírica dos ICs acima:
\begin{enumerate}
\item Fixe um valor de $\theta$ (por ex., $\theta=\widehat\theta_{MV}=2.7027$) e tamanhos $n\in\{6,10,20,30,50\}$.
\item Para cada par $(\theta,n)$, repita $B$ vezes (ex.: $B=10{,}000$):
\begin{enumerate}
\item Gere $Z_1,\ldots,Z_n\overset{iid}\sim \mathrm{Exp}(\theta)$.
\item Calcule $\widehat\theta_{MV}$, as estimativas $\widehat g$ e os ICs (delta) de (a)--(c).
\item Registre se o verdadeiro $g(\theta)$ caiu dentro do IC.
\end{enumerate}
\item Estime a cobertura como a frequência relativa de acertos. Compare com 95\%.
\item Se a cobertura ficar abaixo de 95\%, aumente $n$ até estabilizar próximo de 95\%.
\end{enumerate}
\textit{Expectativa:} Para $n=6$, (a) e (c) tendem a cobertura abaixo de 95\% (assimetria/limites fora do espaço).
Cobertura melhora sensivelmente para $n\gtrsim 30$. 

    \end{answer}

\textbf{Questão 03:}  
Seja $(Z_1, \ldots, Z_n)$ uma amostra aleatória de $Z \sim N(\mu, \sigma^2)$, $\theta = (\mu, \sigma^2) \in (-\infty,\infty) \times (0,\infty)$.
\begin{itemize}
  \item[(a)] Encontre o EMV para $g(\theta) = E_\theta(Z)$.
  \item[(b)] Encontre o EMV para $g(\theta) = P_\theta(Z < 2)$.
  \item[(c)] Encontre o EMV para $g(\theta) = P_\theta(2.6 < Z < 4)$.
  \item[(d)] Encontre o EMV para $g(\theta) = \mathrm{Var}_\theta(Z)$.
  \item[(e)] Considere que os dados foram observados (2.4, 2.7, 2.3, 2, 2.5, 2.6). Encontre as estimativas de MV nos itens acima.
\end{itemize}

    \begin{answer}[]
\textbf{Questão 03.} Seja $(Z_1,\ldots,Z_n)$ uma amostra aleatória de
$Z\sim\mathcal N(\mu,\sigma^2)$, com parâmetro vetorial
$\theta=(\mu,\sigma^2)\in\mathbb R\times(0,\infty)$.

\medskip
\textbf{EMV de $(\mu,\sigma^2)$.}
A densidade conjunta é
\[
L(\mu,\sigma^2;\mathbf z)=
\prod_{i=1}^n \frac{1}{\sqrt{2\pi\sigma^2}}
\exp\!\left\{-\frac{(z_i-\mu)^2}{2\sigma^2}\right\}.
\]
A log-verossimilhança (ignorando constantes que não dependem de $\mu,\sigma^2$) é
\[
\ell(\mu,\sigma^2)
= -\frac{n}{2}\log\sigma^2
  -\frac{1}{2\sigma^2}\sum_{i=1}^n (z_i-\mu)^2 .
\]
Derivando e igualando a zero:
\[
\frac{\partial \ell}{\partial \mu}
= -\frac{1}{\sigma^2}\sum_{i=1}^n (z_i-\mu)
= -\frac{n}{\sigma^2}\,(\bar z-\mu)=0
\;\Longrightarrow\;
\widehat\mu_{MV}=\bar Z=\frac{1}{n}\sum_{i=1}^n Z_i.
\]
Para $\sigma^2$,
\[
\frac{\partial \ell}{\partial \sigma^2}
= -\frac{n}{2}\,\frac{1}{\sigma^2}
  +\frac{1}{2(\sigma^2)^2}\sum_{i=1}^n (z_i-\mu)^2
  =0
\;\Longrightarrow\;
\widehat\sigma^2_{MV}
=\frac{1}{n}\sum_{i=1}^n (Z_i-\bar Z)^2.
\]
A matriz Hessiana é negativa definida em $(\bar z,\widehat\sigma^2)$, de modo que os pontos críticos são máximos globais.

\medskip
\textbf{Princípio de invariância do EMV.}
Para qualquer função $g(\mu,\sigma^2)$, o EMV é o \emph{plug-in}
\[
\widehat g \;=\; g\big(\widehat\mu_{MV},\widehat\sigma^2_{MV}\big).
\]

\bigskip
\textbf{(a) } $g(\theta)=\mathbb E_\theta(Z)=\mu$.
Pela invariância,
\[
\boxed{\;\widehat g_{(a)}=\widehat\mu_{MV}=\bar Z\; }.
\]

\bigskip
\textbf{(b) } $g(\theta)=P_\theta(Z<2)$.
Como $Z\sim\mathcal N(\mu,\sigma^2)$,
\[
P_\theta(Z<2)=\Phi\!\left(\frac{2-\mu}{\sigma}\right),
\]
onde $\Phi$ é a cdf da Normal padrão. Logo,
\[
\boxed{\;\widehat g_{(b)}
=\Phi\!\left(\dfrac{2-\widehat\mu_{MV}}{\widehat\sigma_{MV}}\right)\; }.
\]

\bigskip
\textbf{(c) } $g(\theta)=P_\theta(2.6<Z<4)$.
\[
P_\theta(2.6<Z<4)=
\Phi\!\left(\frac{4-\mu}{\sigma}\right)-
\Phi\!\left(\frac{2.6-\mu}{\sigma}\right),
\]
portanto
\[
\boxed{\;\widehat g_{(c)}
=\Phi\!\left(\dfrac{4-\widehat\mu_{MV}}{\widehat\sigma_{MV}}\right)-
 \Phi\!\left(\dfrac{2.6-\widehat\mu_{MV}}{\widehat\sigma_{MV}}\right)\; }.
\]

\bigskip
\textbf{(d) } $g(\theta)=\mathrm{Var}_\theta(Z)=\sigma^2$.
Pela invariância,
\[
\boxed{\;\widehat g_{(d)}=\widehat\sigma^2_{MV}
=\dfrac{1}{n}\sum_{i=1}^n (Z_i-\bar Z)^2\; }.
\]

\bigskip
\textbf{(e) } Dados observados: $(2.4, 2.7, 2.3, 2.0, 2.5, 2.6)$.
Temos $n=6$, $\sum z_i=14.5$, logo
\[
\bar Z=\frac{14.5}{6}=2.416\overline{6}.
\]
Os desvios ao quadrado:
\[
\sum_{i=1}^n (z_i-\bar Z)^2
=0.00278+0.08028+0.01389+0.17361+0.00694+0.03361
=0.31111\;(\text{aprox.})
\]
Assim,
\[
\widehat\sigma^2_{MV}=\frac{0.31111}{6}=0.05185,\qquad
\widehat\sigma_{MV}=\sqrt{0.05185}\approx 0.2278.
\]

Portanto:
\[
\widehat g_{(a)}=\bar Z\approx 2.4167.
\]
\[
\widehat g_{(b)}=
\Phi\!\left(\frac{2-2.4167}{0.2278}\right)
=\Phi(-1.83)\approx 0.033.
\]
\[
\widehat g_{(c)}=
\Phi\!\left(\frac{4-2.4167}{0.2278}\right)-
\Phi\!\left(\frac{2.6-2.4167}{0.2278}\right)
\approx \Phi(6.95)-\Phi(0.80)
\approx 1-0.2119
\approx 0.788.
\]
\[
\widehat g_{(d)}=\widehat\sigma^2_{MV}\approx 0.0519.
\]

\medskip
\textit{Observação:} $\widehat\sigma^2_{MV}$ usa $1/n$ (EMV). O estimador não-viesado
usa $1/(n-1)$ e não deve ser usado aqui.
    \end{answer}

\textbf{Questão 04:}  
Seja $(Z_1, \ldots, Z_n)$ uma amostra aleatória de $Z \sim f_\theta$, $\theta \in (0,\infty)$, tal que a função densidade de probabilidade é dada por  
\[
f_\theta(x) = \theta \, x^{\theta-1}, \quad x \in (0,1),
\]
e $f_\theta(x) = 0$, caso contrário.
\begin{itemize}
  \item[(a)] Encontre o EMV para $g(\theta) = E_\theta(Z)$.
  \item[(b)] Encontre o EMV para $g(\theta) = P_\theta(Z > 0.3)$.
  \item[(c)] Encontre o EMV para $g(\theta) = P_\theta(0 < Z < 0.1)$.
  \item[(d)] Encontre o EMV para $g(\theta) = \mathrm{Var}_\theta(Z)$.
  \item[(e)] Considere que os dados foram observados (0.12, 0.50, 0.20, 0.23, 0.30, 0.11). Encontre as estimativas de MV para os itens acima.
\end{itemize}

    \begin{answer}[]
\textbf{Questão 04.} Seja $(Z_1,\ldots,Z_n)$ uma amostra i.i.d. de densidade
\[
f_\theta(x)=\theta\,x^{\theta-1},\qquad x\in(0,1),\ \ \theta>0,
\]
e $f_\theta(x)=0$ caso contrário. (Trata-se de uma Beta$(\theta,1)$.)

%---------------------------------------------------
\paragraph{EMV de $\theta$.}
A verossimilhança é
\[
L(\theta;\mathbf z)=\prod_{i=1}^n \theta\,z_i^{\theta-1}
=\theta^n\prod_{i=1}^n z_i^{\theta-1}
=\theta^n\Big(\prod_{i=1}^n z_i\Big)^{\theta-1}.
\]
A log-verossimilhança é
\[
\ell(\theta)=\log L(\theta;\mathbf z)
=n\log\theta+(\theta-1)\sum_{i=1}^n\log z_i .
\]
Derivando e igualando a zero,
\[
\ell'(\theta)=\frac{n}{\theta}+\sum_{i=1}^n\log z_i=0
\quad\Longrightarrow\quad
\widehat\theta_{MV}=-\,\frac{n}{\sum_{i=1}^n\log Z_i}.
\]
A segunda derivada é
\[
\ell''(\theta)=-\,\frac{n}{\theta^2}<0\quad(\theta>0),
\]
logo o ponto crítico é máximo global. Assim,
\[
\boxed{\ \widehat\theta_{MV}=-\,\dfrac{n}{\sum_{i=1}^n\log Z_i}\ }.
\]

%---------------------------------------------------
\paragraph{Momentos e probabilidades.}
Para $X\sim f_\theta$, para $a>- \theta$,
\[
\mathbb E_\theta(X^a)=\int_0^1 x^a\,\theta x^{\theta-1}dx
=\theta\int_0^1 x^{a+\theta-1}dx
=\frac{\theta}{a+\theta}.
\]
Em particular,
\[
\mathbb E_\theta(X)=\frac{\theta}{\theta+1},\qquad
\mathbb E_\theta(X^2)=\frac{\theta}{\theta+2},\qquad
\mathrm{Var}_\theta(X)=\frac{\theta}{(\theta+1)^2(\theta+2)}.
\]
Para $0<a<b\le1$,
\[
P_\theta(a<X<b)=\int_a^b \theta x^{\theta-1}dx
=\Big[x^\theta\Big]_a^b=b^\theta-a^\theta.
\]

%---------------------------------------------------
\paragraph{(a) $g(\theta)=\mathbb E_\theta(Z)$.}
\[
\mathbb E_\theta(Z)=\frac{\theta}{\theta+1}
\quad\Longrightarrow\quad
\boxed{\ \widehat g_{(a)}=\frac{\widehat\theta_{MV}}{\widehat\theta_{MV}+1}\ }.
\]

\paragraph{(b) $g(\theta)=P_\theta(Z>0.3)$.}
\[
P_\theta(Z>0.3)=1-P_\theta(0<Z\le0.3)=1-(0.3)^\theta
\quad\Longrightarrow\quad
\boxed{\ \widehat g_{(b)}=1-(0.3)^{\widehat\theta_{MV}}\ }.
\]

\paragraph{(c) $g(\theta)=P_\theta(0<Z<0.1)$.}
\[
P_\theta(0<Z<0.1)=(0.1)^\theta
\quad\Longrightarrow\quad
\boxed{\ \widehat g_{(c)}=(0.1)^{\widehat\theta_{MV}}\ }.
\]

\paragraph{(d) $g(\theta)=\mathrm{Var}_\theta(Z)$.}
\[
\mathrm{Var}_\theta(Z)=\frac{\theta}{(\theta+1)^2(\theta+2)}
\quad\Longrightarrow\quad
\boxed{\ \widehat g_{(d)}=\frac{\widehat\theta_{MV}}
{(\widehat\theta_{MV}+1)^2(\widehat\theta_{MV}+2)}\ }.
\]

%---------------------------------------------------
\paragraph{(e) Estimativas numéricas.}
Dados: $(0.12,\,0.50,\,0.20,\,0.23,\,0.30,\,0.11)$.
Com $n=6$,
\[
\sum_{i=1}^n\log z_i
=\log(0.12)+\log(0.50)+\log(0.20)+\log(0.23)+\log(0.30)+\log(0.11)
\approx -9.3038.
\]
Logo,
\[
\widehat\theta_{MV}
=-\frac{6}{-9.3038}\approx 0.6449.
\]
Então,
\[
\widehat g_{(a)}=\frac{\widehat\theta}{1+\widehat\theta}
\approx \frac{0.6449}{1.6449}\approx 0.3921,
\]
\[
\widehat g_{(b)}=1-(0.3)^{\widehat\theta}
\approx 1-(0.3)^{0.6449}\approx 0.5400,
\]
\[
\widehat g_{(c)}=(0.1)^{\widehat\theta}
\approx (0.1)^{0.6449}\approx 0.2265,
\]
\[
\widehat g_{(d)}=\frac{\widehat\theta}{(\widehat\theta+1)^2(\widehat\theta+2)}
\approx \frac{0.6449}{(1.6449)^2(2.6449)}\approx 0.0901.
\]

\medskip
\noindent
\textit{Resumo:}\;
$\widehat\theta_{MV}=-n/\sum\log Z_i\approx 0.6449$ e,
por invariância, as estimativas de (a)--(d) são os valores de $g(\theta)$
avaliados em $\widehat\theta$ como mostrado acima.
    \end{answer}

\end{document}
