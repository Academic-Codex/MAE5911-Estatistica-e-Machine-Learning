\paragraph{(a) \(g(\theta)=P_\theta(Z>1)\).}

\begin{enumerate}
  \item Pela fórmula da cauda,
  \[
  g(\theta)=P_\theta(Z>1)=e^{-\theta\cdot 1}=e^{-\theta}.
  \]
  \item Pela \textbf{invariância do EMV},
  \[
  \widehat g_{MV}=g(\widehat\theta_{MV})=e^{-\widehat\theta_{MV}}.
  \]
  \item Usando \(\widehat\theta_{MV}=\dfrac{1}{\overline Z}\),
  \[
  \boxed{\;\widehat g_{(a)}=e^{-\widehat\theta_{MV}}
        =\exp\!\Big(-\frac{1}{\overline Z}\Big).\;}
  \]
\end{enumerate}

\bigskip
\paragraph{(b) \(g(\theta)=P_\theta(0.1<Z<1)\).}

\begin{enumerate}
  \item Para \(0<a<b\), utilizando a expressão para função de distribuição acumulada já calculada,
  \[
  P_\theta(a<Z<b)=F(b)-F(a)=(1-e^{-\theta b})-(1-e^{-\theta a})
                 =e^{-\theta a}-e^{-\theta b}.
  \]
  \item Com \(a=0.1\) e \(b=1\),
  \[
  g(\theta)=e^{-0.1\,\theta}-e^{-\theta}.
  \]
  \item Pela invariância,
  \[
  \;\widehat g_{(b)}=e^{-0.1\,\widehat\theta_{MV}}-e^{-\widehat\theta_{MV}}
        =\exp\!\Big(-\frac{0.1}{\overline Z}\Big)-\exp\!\Big(-\frac{1}{\overline Z}\Big).\;
  \]
  \[
  \boxed{\;\widehat g_{(b)}=
        \exp\!\Big(-\frac{0.1}{\overline Z}\Big)-\exp\!\Big(-\frac{1}{\overline Z}\Big).\;}
  \]
\end{enumerate}

\bigskip
\paragraph{(c) \(g(\theta)=\mathrm{Var}_\theta(Z)\).}

\begin{enumerate}
  \item Para \(Z\sim\mathrm{Exp}(\theta)\), \(E(Z)=\dfrac{1}{\theta}\) e
        \(E(Z^2)=\displaystyle\int_0^\infty z^2\theta e^{-\theta z}dz
                 =\frac{2}{\theta^2}\).
        Assim,
        \[
        \mathrm{Var}_\theta(Z)=E(Z^2)-E(Z)^2=\frac{2}{\theta^2}-\frac{1}{\theta^2}
        =\frac{1}{\theta^2}.
        \]
  \item Logo \(g(\theta)=\theta^{-2}\) e, por invariância,
        \[
        \widehat g_{MV}=g(\widehat\theta_{MV})=\frac{1}{\widehat\theta_{MV}^{\,2}}.
        \]
  \item Como \(\widehat\theta_{MV}=1/\overline Z\),
        \[
        \boxed{\;\widehat g_{(c)}=\frac{1}{(1/\overline Z)^2}=\overline Z^{\,2}. \;}
        \]
\end{enumerate}


\bigskip
\textbf{(d) ICs aproximados de 95\% para $g(\theta)$ (usando o método delta).}
Para a Exponencial, a informação de Fisher é $I(\theta)=\tfrac{n}{\theta^2}$,
de modo que
\[
\widehat\theta_{MV}\ \dot\sim\ N\!\left(\theta,\ \frac{\theta^2}{n}\right).
\]
Pelo método delta, para $g$ diferenciável:
\[
\widehat g\ \dot\sim\ N\!\left(g(\theta),\ \frac{\theta^2}{n}\,[g'(\theta)]^2\right),
\quad\text{e usamos } \theta\leftarrow\widehat\theta_{MV}.
\]
Assim:

\smallskip
\begin{itemize}
\item[(a)] $g(\theta)=e^{-\theta}$, $g'(\theta)=-e^{-\theta}$. Então
\[
\widehat{\mathrm{Var}}(\widehat g_{(a)})
=\frac{\widehat\theta_{MV}^2}{n}\,e^{-2\widehat\theta_{MV}},
\qquad
\text{IC}_{95\%}:\ \widehat g_{(a)}\ \pm\ 1.96\,
\sqrt{\frac{\widehat\theta_{MV}^2}{n}\,e^{-2\widehat\theta_{MV}}}.
\]
\item[(b)] $g(\theta)=e^{-0.1\theta}-e^{-\theta}$,
\(
g'(\theta)=-0.1\,e^{-0.1\theta}+e^{-\theta}.
\)
Então
\[
\widehat{\mathrm{Var}}(\widehat g_{(b)})
=\frac{\widehat\theta_{MV}^2}{n}\,\bigl[-0.1\,e^{-0.1\widehat\theta_{MV}}+e^{-\widehat\theta_{MV}}\bigr]^2,
\]
\[
\text{IC}_{95\%}:\ \widehat g_{(b)}\ \pm\ 1.96\,
\sqrt{\frac{\widehat\theta_{MV}^2}{n}\,\bigl[-0.1\,e^{-0.1\widehat\theta_{MV}}+e^{-\widehat\theta_{MV}}\bigr]^2}.
\]
\item[(c)] $g(\theta)=1/\theta^2$, $g'(\theta)=-2/\theta^3$. Então
\[
\widehat{\mathrm{Var}}(\widehat g_{(c)})
=\frac{4}{n\,\widehat\theta_{MV}^{\,4}},
\qquad
\text{IC}_{95\%}:\ \widehat g_{(c)}\ \pm\ 1.96\,\sqrt{\frac{4}{n\,\widehat\theta_{MV}^{\,4}}}.
\]
\end{itemize}

\medskip
\textbf{Aplicando aos dados} $(0.2,0.6,0.3,0.2,0.8,0.12)$:
\[
n=6,\quad \textstyle\sum z_i=2.22,\quad \overline Z=0.37,\quad
\widehat\theta_{MV}=\frac{n}{\sum z_i}=\frac{6}{2.22}\approx 2.7027.
\]
\emph{Estimativas plug-in:}
\[
\widehat g_{(a)}=e^{-\widehat\theta_{MV}}\approx 0.0668,\qquad
\widehat g_{(b)}=e^{-0.1\widehat\theta_{MV}}-e^{-\widehat\theta_{MV}}\approx 0.6963,\qquad
\widehat g_{(c)}=\overline Z^{\,2}=0.1369.
\]
\emph{ICs (delta, 95\%):}
\[
\text{(a)}\ \ [0,\ 0.211]\ \ (\text{truncado a }[0,1]),
\qquad
\text{(b)}\ \ [0.676,\ 0.717],
\qquad
\text{(c)}\ \ [0,\ 0.356]\ \ (\text{truncado a }[0,\infty)).
\]
\textit{Obs.:} Com $n=6$ o delta pode produzir limites fora do espaço paramétrico; é
padrão truncar aos limites naturais.

\bigskip
\textbf{(e) Cobertura por Monte Carlo (plano de simulação).}
Para verificar a cobertura empírica dos ICs acima:
\begin{enumerate}
\item Fixe um valor de $\theta$ (por ex., $\theta=\widehat\theta_{MV}=2.7027$) e tamanhos $n\in\{6,10,20,30,50\}$.
\item Para cada par $(\theta,n)$, repita $B$ vezes (ex.: $B=10{,}000$):
\begin{enumerate}
\item Gere $Z_1,\ldots,Z_n\overset{iid}\sim \mathrm{Exp}(\theta)$.
\item Calcule $\widehat\theta_{MV}$, as estimativas $\widehat g$ e os ICs (delta) de (a)--(c).
\item Registre se o verdadeiro $g(\theta)$ caiu dentro do IC.
\end{enumerate}
\item Estime a cobertura como a frequência relativa de acertos. Compare com 95\%.
\item Se a cobertura ficar abaixo de 95\%, aumente $n$ até estabilizar próximo de 95\%.
\end{enumerate}
\textit{Expectativa:} Para $n=6$, (a) e (c) tendem a cobertura abaixo de 95\% (assimetria/limites fora do espaço).
Cobertura melhora sensivelmente para $n\gtrsim 30$. 
