\documentclass[a4paper]{article}
\usepackage{student}
\usepackage{graphicx}
\usepackage{caption}
\usepackage[version=4]{mhchem}
\usepackage{tikz}
\usetikzlibrary{shapes.geometric, arrows.meta, positioning, decorations.pathreplacing}
\usepackage{enumitem}
\usepackage[utf8]{inputenc}
\usepackage{amsmath}
\usepackage{booktabs}
\usepackage{float}
\usepackage[portuguese]{babel}
\usepackage[T1]{fontenc}
\usepackage[utf8]{inputenc}
\usepackage[numbers,sort&compress]{natbib} % ou [authoryear]
\usepackage[alf]{abntex2cite} % citação ABNT autor-data
\usepackage{enumerate}
\usepackage{tcolorbox}
\usepackage{listings}
\usepackage{xcolor}

% --- Suporte a UTF-8 ---
\usepackage{lmodern}           % fonte moderna compatível com T1

% --- Pacotes para exibir código ---


% --- Define o idioma R com palavras-chave e cores ---
\lstdefinelanguage{R}{
  keywords={function, if, else, for, in, while, repeat, return, library, require, next, break},
  keywordstyle=\color{blue}\bfseries,
  ndkeywords={TRUE, FALSE, NULL, NA, Inf, NaN},
  ndkeywordstyle=\color{magenta}\bfseries,
  comment=[l]{\#},
  commentstyle=\color{gray}\ttfamily,
  stringstyle=\color{orange}\ttfamily,
  morestring=[b]",
  morestring=[b]',
  sensitive=true
}

% --- Mapeamento para UTF-8 (acentos, ç, etc.) ---
% ---- estilo do listing ----
\lstset{
  literate=
    {á}{{\'a}}1 {ã}{{\~a}}1 {â}{{\^a}}1 {à}{{\`a}}1
    {é}{{\'e}}1 {ê}{{\^e}}1
    {í}{{\'i}}1
    {ó}{{\'o}}1 {õ}{{\~o}}1 {ô}{{\^o}}1
    {ú}{{\'u}}1 {ü}{{\"u}}1
    {ç}{{\c{c}}}1
    {Á}{{\'A}}1 {Ã}{{\~A}}1 {Â}{{\^A}}1
    {É}{{\'E}}1 {Ê}{{\^E}}1
    {Í}{{\'I}}1
    {Ó}{{\'O}}1 {Õ}{{\~O}}1 {Ô}{{\^O}}1
    {Ú}{{\'U}}1 {Ü}{{\"U}}1
    {Ç}{{\c{C}}}1,
  basicstyle=\ttfamily\small,
  backgroundcolor=\color{gray!5},
  frame=single,
  breaklines=true,
  captionpos=b,
  columns=flexible,
  showstringspaces=false,
  keepspaces=true,
  numbers=left,
  numberstyle=\tiny\color{gray},
  stepnumber=1,
  numbersep=6pt
}

\pagestyle{plain}

\tikzstyle{arrow} = [thick,->,>=stealth]

% Definindo o estilo de destaque com linhas pontilhadas
\tikzstyle{highlight} = [draw, dashed, thick, rectangle, rounded corners, inner sep=0.2cm, orange]


\tikzstyle{startstop} = [
    rectangle, rounded corners, minimum width=0.5cm,
    text centered, draw=black, fill=blue!10, font=\small
]
\tikzstyle{startstop_S} = [
    rectangle, rounded corners, minimum width=0.5cm, minimum height=0.8cm,
    text centered, draw=black, fill=green!30, font=\small
]
\tikzstyle{decision} = [
    diamond, aspect=2, draw=black, fill=orange!15, align=center,
    text centered, inner sep=0pt, font=\small
]
\tikzstyle{decision_S} = [
    diamond, aspect=2, draw=black, fill=orange!30, align=center,
    text centered, inner sep=0pt, font=\small
]
\tikzstyle{arrow} = [thick,->,>=stealth]



% Metadata
\date{\today}
\setmodule{MAE5911/IME: Fundamentos de Estatística e Machine Learning. \\ Prof.: Alexandre Galvão Patriota} 
\setterm{2o. semestre, 2025}

%-------------------------------%
% Other details
% TODO: Fill these
%-------------------------------%
\title{Exercício 02 - 15/10}
\setmembername{Nara Avila Moraes}  % Fill group member names
\setmemberuid{5716734}  % Fill group member uids (same order)

%-------------------------------%
% Add / Delete commands and packages
% TODO: Add / Delete here as you need
%-------------------------------%
\usepackage{amsmath,amssymb,bm}

\newcommand{\KL}{\mathrm{KL}}
\newcommand{\R}{\mathbb{R}}
\newcommand{\E}{\mathbb{E}}
\newcommand{\T}{\top}

\newcommand{\expdist}[2]{%
        \normalfont{\textsc{Exp}}(#1, #2)%
    }
\newcommand{\expparam}{\bm \lambda}
\newcommand{\Expparam}{\bm \Lambda}
\newcommand{\natparam}{\bm \eta}
\newcommand{\Natparam}{\bm H}
\newcommand{\sufstat}{\bm u}

% Main document
\begin{document}
% Add header
\header{}

\textbf{Questão 01:}
Seja $(Z_1, \ldots, Z_n)$ uma amostra aleatória de $Z \sim \text{Ber}(\theta)$, $\theta \in (0,1)$.
\begin{itemize}
    \item[(a)] Encontre o EMV para $g(\theta) = P_\theta(Z = 0)$.
    \item[(b)] Encontre o EMV para $g(\theta) = \mathrm{Var}_\theta(Z)$.
    \item[(c)] Considere que os dados foram observados (0, 0, 1, 0, 0, 1). Encontre as estimativas de MV nos itens acima.
    \item[(d)] Construa o valor-p para a hipótese H: ``$\theta = 0.1$'' usando os dados do item anterior.
\end{itemize}

\begin{research}[]
    O estimador via função de máxima verossimilhança foi proposto por Fisher, que demonstrou a superioridade do deste método referente a outros métodos estimadores, como o método de momentos. 
Este estimador é utilizado sobretudo quando a informação da distribuição de probabilidade do modelo é conhecida. 
Ela tem propriedades como eficiência assintótica, consistência, distribuição normal e invariancia [Notas de aula pg. 60]. 
O estimador de máxima verossimilhança representa uma estimativa para o valor do parâmetro $\theta$ que torna os dados observados o mais plausível possível. 
\\[0.5em]
Devido a propriedade de invariancia do EMV, basta encontrarmos EMV para a letra (a) e depois calcular a letra (b) utilizando este estimador como parâmetro.


Seja \( (Z_1, \dots, Z_n) \) a amostra aleatória de 
\[
Z_i \sim \mathrm{Bernoulli}(\theta), \quad \theta \in (0,1).
\]

A função de verossimilhança, que é a função de densidade de probabilidade conjunta do modelo, é dada por
\[
L(\theta; z_1, \dots, z_n)
= \prod_{i=1}^n \theta^{z_i}(1-\theta)^{1-z_i}.
\]

Para encontrar o estimador de máxima verossimilhança, precisamos maximizar \( L(\theta) \) em relação a \( \theta \). Mas como esta função se trata de um produtório,
é mais fácil maximizar a função log-verossimilhança, que por ser monotonicamente crescente, preserva o valor de theta no ponto máximo \( L(\theta) \).


\text{Tomando o logaritmo natural em ambos os lados:}

\[
\ell(\theta)
= \log L(\theta; z_1, \ldots, z_n)
= \log\!\left(\prod_{i=1}^{n} \theta^{z_i}(1-\theta)^{1-z_i}\right).
\]

\text{Sabendo que } 
\(
\log(ab) = \log a + \log b \text{ e que } 
\log\!\left(\prod_i a_i\right) = \sum_i \log a_i, \text{ temos:}
\)

\[
\ell(\theta)
= \sum_{i=1}^{n} \log\!\big(\theta^{z_i}(1-\theta)^{1-z_i}\big).
\]

\text{Aplicando novamente a propriedade do logaritmo de um produto: }

\[
\ell(\theta)
= \sum_{i=1}^{n} \big[\log(\theta^{z_i}) + \log((1-\theta)^{1-z_i})\big].
\]

\(
\text{Sabendo que } \log(a^b) = b\log a, \text{ obtemos:}
\)

\[
\ell(\theta)
= \sum_{i=1}^{n} \big[z_i \log \theta + (1 - z_i)\log(1-\theta)\big].
\]

\text{Por fim, separando as somas:}

\[
\ell(\theta)
= \left(\sum_{i=1}^{n} z_i\right)\log \theta
+ \left(\sum_{i=1}^{n}(1-z_i)\right)\log(1-\theta).
\]

A log-verossimilhança correspondente é
\[
\boxed{\ell(\theta) 
= \sum_{i=1}^n \big[z_i \log\theta + (1-z_i)\log(1-\theta)\big].}
\]

Para derivar em relação a \(\theta\), sabemos que 
\(\dfrac{d}{d\theta}\log\theta=\dfrac1\theta\) e 
\(\dfrac{d}{d\theta}\log(1-\theta)= -\dfrac1{1-\theta}\), obtemos,
termo a termo:

\[
\frac{d}{d\theta}\Big[z_i\log\theta\Big]
= z_i\,\frac1\theta
\qquad\text{e}\qquad
\frac{d}{d\theta}\Big[(1-z_i)\log(1-\theta)\Big]
= (1-z_i)\Big(-\frac1{1-\theta}\Big)
= -\,\frac{1-z_i}{1-\theta}.
\]
Logo,
\[
\ell'(\theta)
=\sum_{i=1}^n\left(\frac{z_i}{\theta}-\frac{1-z_i}{1-\theta}\right)
= \frac{\sum_{i=1}^n z_i}{\theta} - \frac{\sum_{i=1}^n(1-z_i)}{1-\theta}
= \frac{\sum_i z_i}{\theta} - \frac{n-\sum_i z_i}{1-\theta}.
\]

\text{Igualando a zero para encontrar o} \textbf{ponto crítico.}
\[
\ell'(\theta)=0
\;\Longleftrightarrow\;
\frac{\sum_i z_i}{\theta}
= \frac{n-\sum_i z_i}{1-\theta}
\;\Longleftrightarrow\;
(1-\theta)\sum_i z_i=\theta\,(n-\sum_i z_i).
\]
Expandindo e rearranjando:
\[
\sum_i z_i - \theta\sum_i z_i = \theta n - \theta\sum_i z_i
\;\Longleftrightarrow\;
\sum_i z_i = \theta n
\;\Longleftrightarrow\;
\widehat\theta_{MV}=\frac1n\sum_{i=1}^n z_i=\overline Z.
\]


Obtivemos o estimador de máxima verossimilhança para a probabilidade de sucesso \(\theta\):
\[
\boxed{\hat{\theta}_{MV} = \bar{Z} = \frac{1}{n}\sum_{i=1}^n Z_i.}
\]

O teste de hipótese, mede o quão improvável seria observar $\hat{\theta}_{MV}$ se a hipótese nula fosse verdadeira. O valor-p é esta medida de improbabilidade.
\\[1em]
Em condições de regularidade, o valor-$p$ segue uma distribuição uniforme em $(0,1)$.
Isso significa que, se o seu valor é muito pequeno, por exemplo $p \leq 0{,}01$,
então o resultado observado é um evento raro sob a hipótese nula $H_0$,
ou, alternativamente, que a hipótese nula está incorreta.
Nessa situação, existem evidências para rejeitar $H_0$.

\end{research}

\begin{answer}[]
    \paragraph{(a) \(g(\theta)=P_\theta(Z=0)\)}

\[
  g(\theta)=P_\theta(Z=0)=1-\theta.
\]

Pelo teorema da \textbf{invariância do EMV}:
\[
  \widehat g_{MV}=g(\widehat\theta_{MV})=1-\widehat\theta_{MV}.
\]

Já encontramos que \(\widehat\theta_{MV}=\overline Z\), então:
\[
  \boxed{\widehat g_{MV}=1-\overline Z}.
\]



\paragraph{(b) \(g(\theta)=\mathrm{Var}_\theta(Z)\)}

\[\]
Para uma Bernoulli, a variância é dada por:
\[
  \mathrm{Var}(Z)=E(Z^2)-E(Z)^2.
\]

Pela definição de esperança para variáveis discretas,
\[
  E(Z) = \sum_{z \in \{0,1\}} z \, P(Z = z).
\]

Mas $P(Z=1)=\theta$ e $P(Z=0)=1-\theta$, então:
\[
  E(Z)
  = 0 \cdot (1-\theta) + 1 \cdot \theta
  = \theta.
\]

\(E(Z^2)=E(Z)=\theta\) (pois \(Z\in\{0,1\}\)), então:
\[
  \mathrm{Var}_\theta(Z)=\theta-\theta^2=\theta(1-\theta).
\]

Portanto:
\[
  g(\theta)=\mathrm{Var}_\theta(Z)=\theta(1-\theta).
\]
Pela \textbf{invariância do EMV}:
\[
  \widehat g_{MV}=g(\widehat\theta_{MV})
  =\widehat\theta_{MV}(1-\widehat\theta_{MV}).
\]

Para \(\widehat\theta_{MV}=\overline Z\), encontramos:
\[
  \boxed{\widehat g_{MV}=\overline Z(1-\overline Z).}
\]



\paragraph{(c) Sejam os dados observados: \( (0, 0, 1, 0, 0, 1) \)}
\[\]
Para o cálculo da média temos: \( n = 6 \) e \( \sum z_i = 2 \), portanto
\[
  \hat{\theta}_{MV} = \bar{Z} = \frac{2}{6} = \frac{1}{3}.
\]
Assim:
\[
  (a) \quad\hat{g_{MV}}=\widehat{P(Z=0)} = 1 - \hat{\theta}_{MV} = \frac{2}{3},
\]
\[
  (b) \quad\hat{g_{MV}}=\widehat{\mathrm{Var}}(Z) = \hat{\theta}_{MV}(1 - \hat{\theta}_{MV})
  = \frac{1}{3} \cdot \frac{2}{3} = \frac{2}{9}.
\]

---

%%%%%%%%%%%%%%%%%%%%%%%%%%%%%%%%%%%%%%%%%%%%%%%%%%%%%%%%%%%%%%%%%%%%%%%%%%%%%%

\end{answer}


\textbf{Questão 02:}
Seja $(Z_1, \ldots, Z_n)$ uma amostra aleatória de $Z \sim \text{Exp}(\theta)$, $\theta \in (0,\infty)$.
\begin{itemize}
    \item[(a)] Encontre o EMV para $g(\theta) = P_\theta(Z > 1)$.
    \item[(b)] Encontre o EMV para $g(\theta) = P_\theta(0.1 < Z < 1)$.
    \item[(c)] Encontre o EMV para $g(\theta) = \mathrm{Var}_\theta(Z)$.
    \item[(d)] Considere que os dados foram observados (0.2, 0.6, 0.3, 0.2, 0.8, 0.12). Encontre um IC aproximado de 95\% de confiança para $g(\theta)$ nos itens acima.
    \item[(e)] Faça uma simulação de Monte Carlo para verificar se os IC's aproximados obtidos no passo anterior têm cobertura próxima do nível de confiança estabelecido. Caso não tenham, proponha um tamanho amostral que produza IC's mais confiáveis para cada caso.
\end{itemize}

\begin{research}[]
    Podemos calcular os ítens (a), (b) e (c) utilizando a propriedade de invariância do estimador de máxima verossimilhança (EMV), calculando uma única vez $\hat\theta_{MV}$.

\subsection*{Cálculo do EMV de $\theta$ ( $\hat\theta_{MV}$) para a distribuição Exponencial}

\textbf{Função de verossimilhança}

Seja $(z_1,\dots,z_n)$ uma amostra aleatória de $Z \sim \text{Exp}(\theta)$, com $ \theta \in (0,\infty)$, a função densidade de probabilidade para $z_i$ é então:
\[
      f(z_i;\theta) = \theta e^{-\theta z_i}, \qquad z_i>0.
\]

Como $z_1,\dots,z_n$ são independentes, a probabilidade conjunta, definida como a função de verossimilhança é o produto das densidades individuais:
\[
      \ell(\theta;z_1,\dots,z_n)
      = \prod_{i=1}^n f(z_i\mid \theta)
      = \prod_{i=1}^n \theta e^{-\theta z_i} \mathbf 1_{\{z_i>0\}}.
\]

O termo indicador $\prod_{i=1}^n \mathbf 1_{\{z_i>0\}}$ não depende de $\theta$, logo
não influencia a maximização e pode ser ignorado. Assim,
\[
      \ell(\theta;z_1,\dots,z_n)
      = \prod_{i=1}^n \theta e^{-\theta z_i}.
\]

Aplicando a propriedade distributiva do produto e de produto de exponenciais, obtemos:
\[
      \ell(\theta;z_1,\dots,z_n)
      = \biggl(\prod_{i=1}^n \theta\biggr)
      \biggl(\prod_{i=1}^n e^{-\theta z_i}\biggr)
      = \theta^n \exp\!\left(-\theta\sum_{i=1}^n z_i\right).
\]

Seja
\[
      S = \sum_{i=1}^n z_i,
\]
Podemos escrever:
\[
      \ell(\theta,S) = \theta^n \exp(-\theta S).
\]

\textbf{Função log-verossimilhança}

Para simplificar os cálculos, trabalhamos com a função logaritmo da verossimilhança $\mathcal{L}(\theta, S)$:
\[
      \mathcal{L}(\theta, S) = \log \ell(\theta,S)
      = \log(\theta^n) + \log\!\left(\exp(-\theta S)\right).
\]

Usando as propriedades $\log(a^b)=b\log a$ e $\log(e^x)=x$, obtemos
\[
      \mathcal{L}(\theta, S) = n\log\theta - \theta S.
\]

\textbf{Ponto crítico da log-verossimilhança}

Calculando a derivada de $\mathcal{L}(\theta, S)$ em relação a $\theta$:
\[
      \frac{\partial\mathcal{L}(\theta, S)}{\partial\theta}
      = \frac{\partial}{\partial\theta}\bigl[n\log\theta - \theta S\bigr]
      = n\cdot\frac{1}{\theta} - S\cdot 1
      = \frac{n}{\theta} - S.
\]

Seja a condição de máximo onde está definido $\hat\theta_{MV}$:
\[
      \frac{\partial\mathcal{L}(\theta, S)}{\partial\theta} = 0
      \iff
      \frac{n}{\hat\theta_{MV}} - S = 0
      \iff
      \frac{n}{\hat\theta_{MV}} = S.
\]

Isolando $\hat\theta_{MV}$:

\[
      \boxed{
            \hat\theta_{MV}
            = \frac{n}{S},
      }
\]

\textbf{Estimador de máxima verossimilhança}

Substituindo de volta em funcão da variável aleatória $z_i$, obtemos:
\[
      \hat\theta_{MV}
      = \frac{n}{\sum_{i=1}^n z_i}
      = \frac{1}{\bar Z},
\]


\subsection*{Intervalo de confiança}

Conhecendo a \textbf{distribuição de probabilidade de} $\hat\theta$ é possível construir um intervalo $$\hat\theta_1 \leq \theta \leq \hat\theta_2$$ que contém $\theta$.
Esta técnica diferencia-se da estimação "por ponto", onde se calcula um único valor para o parâmetro populacional. No caso do intervalo de confiança busca-se um
seguimento, ou intervalo $\hat\theta_1:\hat\theta_2$ que contém o parâmetro desconhecido (FONSECA, 2008, p. 186).


% \paragraph{Composição da distribuição Exponencial contínua acumulada.}
% \vspace{1em}
% Separando a distribuição exponencial em função de distribuição acumulada até a observação de interesse e
% o restante da distribuição acumulada denominada a cauda da distribuição, temos:
% \\[1em]
% Expressão para densidade total: \[f(z_i\mid\theta)=\theta e^{-\theta z_i}\mathbf 1_{\{z_i\ge0\}}.\] \\
% Função de distribuição acumulada: \[F(z)=P_\theta(Z\le z)=\int_0^z \theta e^{-\theta t}\,dt
%       =\big[-e^{-\theta t}\big]_0^z=1-e^{-\theta z}.\] \\
% Cauda: \[P_\theta(Z>z)=1-F(z)=e^{-\theta z}.\]
\end{research}

\begin{answer}[]
    Seja $Z_1,\ldots,Z_n \stackrel{\text{iid}}{\sim}\mathrm{Exp}(\theta)$, com $\theta>0$.
A função densidade de probabilidade é
\[
f(z\mid\theta)=\theta\,e^{-\theta z}\,\mathbf 1_{\{z\ge 0\}},
\]
onde $\mathbf 1_{\{z\ge 0\}}$ é a função indicadora (vale 1 se $z\ge 0$ e 0 caso contrário).

Como as observações são independentes e identicamente distribuídas,
a \textbf{função de verossimilhança}, que é a função de densidade de probabilidade conjunta, é o produto das densidades individuais:
\[
L(\theta;z_1,\ldots,z_n)
=\prod_{i=1}^n f(z_i\mid\theta)
=\prod_{i=1}^n \theta\,e^{-\theta z_i}\,\mathbf 1_{\{z_i\ge 0\}}.
\]

Aplicando a propriedade distributiva do produto, separamos os fatores:
\[
L(\theta;z)
=\Big(\prod_{i=1}^n \mathbf 1_{\{z_i\ge 0\}}\Big)
\Big(\prod_{i=1}^n \theta e^{-\theta z_i}\Big).
\]

Como o termo indicador $\prod \mathbf 1_{\{z_i\ge 0\}}$ não depende de $\theta$,
ele não influencia a maximização e pode ser ignorado. Assim:
\[
L(\theta;z)
=\prod_{i=1}^n (\theta e^{-\theta z_i}).
\]

Usando a propriedade $\prod(ab)=\prod a \cdot \prod b$, obtemos:
\[
L(\theta;z)
=\Big(\prod_{i=1}^n \theta\Big)
\Big(\prod_{i=1}^n e^{-\theta z_i}\Big).
\]

Aplicando:
1. $\prod_{i=1}^n \theta = \theta^n$ (produto de n fatores iguais),  
2. $\prod e^{a_i} = e^{\sum a_i}$ (exponencial do somatório),  segue que a função de verossimilhança é da forma:
\[
\boxed{
L(\theta;z)
=\theta^n \exp\!\left(-\theta \sum_{i=1}^n z_i\right).}
\]

---

Para encontrar a função \textbf{Log-verossimilhança}, aplicamos $\log$ em ambos os lados, e usamos as propriedades do logaritmo:

1. $\log(ab)=\log a+\log b$  
2. $\log(a^b)=b\log a$  
3. $\log(e^x)=x$  
4. $\log(\prod a_i)=\sum \log a_i$

\[
\ell(\theta)=\log L(\theta;z)
=\log(\theta^n)+\log\!\left(\exp\!\left[-\theta\sum_{i=1}^n z_i\right]\right).
\]

Aplicando $\log(a^b)=b\log a$ no primeiro termo
e $\log(e^x)=x$ no segundo termo, obtemos:
\[
\ell(\theta)=n\log\theta-\theta\sum_{i=1}^n z_i.
\]

Para maximizar, derivamos em relação a $\theta$, aplicando as regras de derivação:
\[
\frac{d}{d\theta}\log\theta=\frac{1}{\theta}, \qquad
\frac{d}{d\theta}(a\theta)=a.
\]

\[
\ell'(\theta)
=\frac{d}{d\theta}[n\log\theta-\theta\sum_{i=1}^n z_i]
=n\frac{1}{\theta}-\sum_{i=1}^n z_i
=\frac{n}{\theta}-\sum_{i=1}^n z_i.
\]

Logo:
\[
\boxed{
\ell'(\theta)
=\frac{n}{\theta}-\sum_{i=1}^n z_i.}
\]

Igualamos a zero para encontrar o \textbf{ponto crítico} que maximiza a função. 
\[
\ell'(\theta)=0
\;\Longleftrightarrow\;
\frac{n}{\theta}-\sum_{i=1}^n z_i=0
\;\Longleftrightarrow\;
\frac{n}{\theta}=\sum_{i=1}^n z_i
\;\Longleftrightarrow\;
\widehat\theta_{MV}=\frac{n}{\sum_{i=1}^n z_i}
=\frac{1}{\overline Z}.
\]

---

Para assegurar que se trata de um ponto de máximo, derivamos novamente:

\[
\ell''(\theta)
=\frac{d}{d\theta}\!\left(\frac{n}{\theta}-\sum_{i=1}^n z_i\right)
=-\frac{n}{\theta^2}.
\]

Como $\ell''(\theta)<0$ para $\theta>0$,
o ponto crítico corresponde a um máximo.

---

O \textbf{estimador de máxima verossimilhança} (EMV) é, portanto:
\[
\boxed{\widehat\theta_{MV}=\frac{n}{\sum_{i=1}^n z_i}
=\frac{1}{\overline Z}},
\]

\medskip
É válido a propriedade da \textbf{Invariância do EMV}: para $g(\theta)$, o EMV é $\widehat g=g(\widehat\theta_{MV})$.

\paragraph{Composição da distribuição Exponencial contínua acumulada.} \newline
\vspace{1em}
Separando a distribuição exponencial em função de distribuição acumulada até a observação de interesse e 
o restante da distribuição acumulada denominada a cauda da distribuição, temos:
 \\[1em]
Expressão para densidade total: \[f(z\mid\theta)=\theta e^{-\theta z}\mathbf 1_{\{z\ge0\}}\]. \\
Função de distribuição acumulada: \[F(z)=P_\theta(Z\le z)=\int_0^z \theta e^{-\theta t}\,dt
      =\big[-e^{-\theta t}\big]_0^z=1-e^{-\theta z}\]. \\
Cauda: \[P_\theta(Z>z)=1-F(z)=e^{-\theta z}\].

\bigskip
\paragraph{(a) \(g(\theta)=P_\theta(Z>1)\).}

\begin{enumerate}
  \item Pela fórmula da cauda,
  \[
  g(\theta)=P_\theta(Z>1)=e^{-\theta\cdot 1}=e^{-\theta}.
  \]
  \item Pela \textbf{invariância do EMV},
  \[
  \widehat g_{MV}=g(\widehat\theta_{MV})=e^{-\widehat\theta_{MV}}.
  \]
  \item Usando \(\widehat\theta_{MV}=\dfrac{1}{\overline Z}\),
  \[
  \boxed{\;\widehat g_{(a)}=e^{-\widehat\theta_{MV}}
        =\exp\!\Big(-\frac{1}{\overline Z}\Big).\;}
  \]
\end{enumerate}

\bigskip
\paragraph{(b) \(g(\theta)=P_\theta(0.1<Z<1)\).}

\begin{enumerate}
  \item Para \(0<a<b\), utilizando a expressão para função de distribuição acumulada já calculada,
  \[
  P_\theta(a<Z<b)=F(b)-F(a)=(1-e^{-\theta b})-(1-e^{-\theta a})
                 =e^{-\theta a}-e^{-\theta b}.
  \]
  \item Com \(a=0.1\) e \(b=1\),
  \[
  g(\theta)=e^{-0.1\,\theta}-e^{-\theta}.
  \]
  \item Pela invariância,
  \[
  \;\widehat g_{(b)}=e^{-0.1\,\widehat\theta_{MV}}-e^{-\widehat\theta_{MV}}
        =\exp\!\Big(-\frac{0.1}{\overline Z}\Big)-\exp\!\Big(-\frac{1}{\overline Z}\Big).\;
  \]
  \[
  \boxed{\;\widehat g_{(b)}=
        \exp\!\Big(-\frac{0.1}{\overline Z}\Big)-\exp\!\Big(-\frac{1}{\overline Z}\Big).\;}
  \]
\end{enumerate}

\bigskip
\paragraph{(c) \(g(\theta)=\mathrm{Var}_\theta(Z)\).}

\begin{enumerate}
  \item Para \(Z\sim\mathrm{Exp}(\theta)\), \(E(Z)=\dfrac{1}{\theta}\) e
        \(E(Z^2)=\displaystyle\int_0^\infty z^2\theta e^{-\theta z}dz
                 =\frac{2}{\theta^2}\).
        Assim,
        \[
        \mathrm{Var}_\theta(Z)=E(Z^2)-E(Z)^2=\frac{2}{\theta^2}-\frac{1}{\theta^2}
        =\frac{1}{\theta^2}.
        \]
  \item Logo \(g(\theta)=\theta^{-2}\) e, por invariância,
        \[
        \widehat g_{MV}=g(\widehat\theta_{MV})=\frac{1}{\widehat\theta_{MV}^{\,2}}.
        \]
  \item Como \(\widehat\theta_{MV}=1/\overline Z\),
        \[
        \boxed{\;\widehat g_{(c)}=\frac{1}{(1/\overline Z)^2}=\overline Z^{\,2}. \;}
        \]
\end{enumerate}


\bigskip
\textbf{(d) ICs aproximados de 95\% para $g(\theta)$ (usando o método delta).}
Para a Exponencial, a informação de Fisher é $I(\theta)=\tfrac{n}{\theta^2}$,
de modo que
\[
\widehat\theta_{MV}\ \dot\sim\ N\!\left(\theta,\ \frac{\theta^2}{n}\right).
\]
Pelo método delta, para $g$ diferenciável:
\[
\widehat g\ \dot\sim\ N\!\left(g(\theta),\ \frac{\theta^2}{n}\,[g'(\theta)]^2\right),
\quad\text{e usamos } \theta\leftarrow\widehat\theta_{MV}.
\]
Assim:

\smallskip
\begin{itemize}
\item[(a)] $g(\theta)=e^{-\theta}$, $g'(\theta)=-e^{-\theta}$. Então
\[
\widehat{\mathrm{Var}}(\widehat g_{(a)})
=\frac{\widehat\theta_{MV}^2}{n}\,e^{-2\widehat\theta_{MV}},
\qquad
\text{IC}_{95\%}:\ \widehat g_{(a)}\ \pm\ 1.96\,
\sqrt{\frac{\widehat\theta_{MV}^2}{n}\,e^{-2\widehat\theta_{MV}}}.
\]
\item[(b)] $g(\theta)=e^{-0.1\theta}-e^{-\theta}$,
\(
g'(\theta)=-0.1\,e^{-0.1\theta}+e^{-\theta}.
\)
Então
\[
\widehat{\mathrm{Var}}(\widehat g_{(b)})
=\frac{\widehat\theta_{MV}^2}{n}\,\bigl[-0.1\,e^{-0.1\widehat\theta_{MV}}+e^{-\widehat\theta_{MV}}\bigr]^2,
\]
\[
\text{IC}_{95\%}:\ \widehat g_{(b)}\ \pm\ 1.96\,
\sqrt{\frac{\widehat\theta_{MV}^2}{n}\,\bigl[-0.1\,e^{-0.1\widehat\theta_{MV}}+e^{-\widehat\theta_{MV}}\bigr]^2}.
\]
\item[(c)] $g(\theta)=1/\theta^2$, $g'(\theta)=-2/\theta^3$. Então
\[
\widehat{\mathrm{Var}}(\widehat g_{(c)})
=\frac{4}{n\,\widehat\theta_{MV}^{\,4}},
\qquad
\text{IC}_{95\%}:\ \widehat g_{(c)}\ \pm\ 1.96\,\sqrt{\frac{4}{n\,\widehat\theta_{MV}^{\,4}}}.
\]
\end{itemize}

\medskip
\textbf{Aplicando aos dados} $(0.2,0.6,0.3,0.2,0.8,0.12)$:
\[
n=6,\quad \textstyle\sum z_i=2.22,\quad \overline Z=0.37,\quad
\widehat\theta_{MV}=\frac{n}{\sum z_i}=\frac{6}{2.22}\approx 2.7027.
\]
\emph{Estimativas plug-in:}
\[
\widehat g_{(a)}=e^{-\widehat\theta_{MV}}\approx 0.0668,\qquad
\widehat g_{(b)}=e^{-0.1\widehat\theta_{MV}}-e^{-\widehat\theta_{MV}}\approx 0.6963,\qquad
\widehat g_{(c)}=\overline Z^{\,2}=0.1369.
\]
\emph{ICs (delta, 95\%):}
\[
\text{(a)}\ \ [0,\ 0.211]\ \ (\text{truncado a }[0,1]),
\qquad
\text{(b)}\ \ [0.676,\ 0.717],
\qquad
\text{(c)}\ \ [0,\ 0.356]\ \ (\text{truncado a }[0,\infty)).
\]
\textit{Obs.:} Com $n=6$ o delta pode produzir limites fora do espaço paramétrico; é
padrão truncar aos limites naturais.

\bigskip
\textbf{(e) Cobertura por Monte Carlo (plano de simulação).}
Para verificar a cobertura empírica dos ICs acima:
\begin{enumerate}
\item Fixe um valor de $\theta$ (por ex., $\theta=\widehat\theta_{MV}=2.7027$) e tamanhos $n\in\{6,10,20,30,50\}$.
\item Para cada par $(\theta,n)$, repita $B$ vezes (ex.: $B=10{,}000$):
\begin{enumerate}
\item Gere $Z_1,\ldots,Z_n\overset{iid}\sim \mathrm{Exp}(\theta)$.
\item Calcule $\widehat\theta_{MV}$, as estimativas $\widehat g$ e os ICs (delta) de (a)--(c).
\item Registre se o verdadeiro $g(\theta)$ caiu dentro do IC.
\end{enumerate}
\item Estime a cobertura como a frequência relativa de acertos. Compare com 95\%.
\item Se a cobertura ficar abaixo de 95\%, aumente $n$ até estabilizar próximo de 95\%.
\end{enumerate}
\textit{Expectativa:} Para $n=6$, (a) e (c) tendem a cobertura abaixo de 95\% (assimetria/limites fora do espaço).
Cobertura melhora sensivelmente para $n\gtrsim 30$. 

\end{answer}
\begin{lstlisting}[language=R, caption={Simulação de Monte Carlo para testar o intervalo de confiança encontrado para a letra (a)}]
set.seed(123)

# Parâmetros do experimento
theta0   <- 2.703     # "verdadeiro" theta
n        <- 6         # tamanho da amostra
M        <- 1000      # número de simulações
z_0975   <- 1.96      # quantil 0.975 da N(0,1)

# valor verdadeiro de g(theta) = P(Z > 1)
g_true <- exp(-theta0)

# vetor para guardar se o IC cobriu ou não
covered <- logical(M)

for (m in 1:M) {
  # 1) cria uma amostra com distribuição Exponencial(theta0)
  z <- rexp(n, theta0)
  
  # 2) EMV de theta: 1 / media
  theta_hat <- 1 / mean(z)
  
  # 3) IC aproximado para theta
  L_theta <- theta_hat * (1 - z_0975 / sqrt(n))
  U_theta <- theta_hat * (1 + z_0975 / sqrt(n))
  
  # 4) transforma para g(theta) = e^{-theta}
  L_g <- exp(-U_theta)
  U_g <- exp(-L_theta)
  
  # 5) verifica se g_true está no intervalo
  covered[m] <- (L_g <= g_true) && (g_true <= U_g)
}

# Estimativa da cobertura
mean(covered)
\end{lstlisting}

\begin{lstlisting}[language=R, caption={Simulação de Monte Carlo para testar o intervalo de confiança encontrado para a letra (b)}]
set.seed(123)

# Parâmetros do experimento
theta0  <- 2.703        # "verdadeiro" theta (mesmo do item a)
n       <- 6            # tamanho da amostra
M       <- 10000        # número de simulações
z_0975  <- 1.96         # quantil 0.975 da N(0,1)

# Função g(theta) do item (b):
# g(theta) = P(0.1 < Z < 1) = exp(-0.1*theta) - exp(-theta)
g_fun  <- function(theta) exp(-0.1 * theta) - exp(-theta)
g_true <- g_fun(theta0)

# Ponto onde g(theta) atinge seu máximo
theta_star <- log(10) / 0.9

covered <- logical(M)

for (m in 1:M) {
  
  # 1) Amostra Z_1,...,Z_n ~ Exponencial(theta0)
  z <- rexp(n, rate = theta0)
  
  # 2) EMV de theta: 1 / media
  theta_hat <- 1 / mean(z)
  
  # 3) IC aproximado para theta
  L_theta <- theta_hat * (1 - z_0975 / sqrt(n))
  U_theta <- theta_hat * (1 + z_0975 / sqrt(n))
  
  # 4) Intervalo para g(theta)
  g_L <- g_fun(L_theta)
  g_U <- g_fun(U_theta)
  
  # Como g(theta) NÂO é monotona, verificamos se o máximo está no intervalo
  if (L_theta <= theta_star && theta_star <= U_theta) {
    g_max <- g_fun(theta_star)
    g_min <- min(g_L, g_U)
  } else {
    g_min <- min(g_L, g_U)
    g_max <- max(g_L, g_U)
  }
  
  # 5) Verifica cobertura
  covered[m] <- (g_min <= g_true) && (g_true <= g_max)
}

# Estimativa da cobertura
mean(covered)
\end{lstlisting}

\textbf{Questão 03:}
Seja $(Z_1, \ldots, Z_n)$ uma amostra aleatória de $Z \sim N(\mu, \sigma^2)$, $\theta = (\mu, \sigma^2) \in (-\infty,\infty) \times (0,\infty)$.
\begin{itemize}
    \item[(a)] Encontre o EMV para $g(\theta) = E_\theta(Z)$.
    \item[(b)] Encontre o EMV para $g(\theta) = P_\theta(Z < 2)$.
    \item[(c)] Encontre o EMV para $g(\theta) = P_\theta(2.6 < Z < 4)$.
    \item[(d)] Encontre o EMV para $g(\theta) = \mathrm{Var}_\theta(Z)$.
    \item[(e)] Considere que os dados foram observados (2.4, 2.7, 2.3, 2, 2.5, 2.6). Encontre as estimativas de MV nos itens acima.
\end{itemize}


\begin{research}[]
    Podemos calcular os ítens (a), (b), (c) e (d) utilizando a propriedade de invariância do estimador de máxima verossimilhança (EMV), calculando uma única vez $\hat\theta_{MV}$.
\subsection*{Cálculo do EMV de \(\theta\) ( \(\hat\theta_{MV}\)) para a distribuição Normal}
\textbf{Função de verossimilhança}

Seja $(z_1,\dots,z_n)$ uma amostra aleatória de $Z \sim N(\mu,\sigma^2)$,
com parâmetro $\theta = (\mu,\sigma^2) \in (-\infty,\infty)\times(0,\infty)$.

A função densidade de probabilidade para $z_i$ é
\[
    f(z_i;\mu,\sigma^2)
    = \frac{1}{\sqrt{2\pi}\,\sigma}
    \exp\!\left(
    -\frac{(z_i-\mu)^2}{2\sigma^2}
    \right).
\]

Como $z_1,\dots,z_n$ são independentes, a função de verossimilhança é
o produto das densidades individuais:
\[
    \ell(\mu,\sigma^2;z_1,\ldots,z_n)
    = \prod_{i=1}^n f(z_i;\mu,\sigma^2).
\]

Aplicando a propriedade distributiva do produto e de produto de exponenciais, obtemos:

\[
    \ell(\mu,\sigma^2;z_1,\ldots,z_n)
    = \left( \frac{1}{\sqrt{2\pi}\,\sigma} \right)^n
    \exp\!\left(
    -\frac{1}{2\sigma^2}\sum_{i=1}^n (z_i-\mu)^2
    \right).
\]

Seja
\[
    S = \sum_{i=1}^n z_i \qquad e \qquad Q = \sum_{i=1}^n z_i^2, \qquad \iff \qquad   \sum_{i=1}^n (z_i-\mu)^2
    = \sum_{i=1}^n (z_i^2 - 2\mu z_i + \mu^2)
    = Q - 2\mu S + n\mu^2.
\]
Podemos escrever:

\[
    \ell(\mu,\sigma^2,S,Q)
    = \left( \frac{1}{\sqrt{2\pi}\,\sigma} \right)^n
    \exp\!\left(
    Q - 2\mu S + n\mu^2
    \right).
\]

Para simplificar os cálculos, trabalhamos com a função logaritmo da verossimilhança $\mathcal{L}(\mu,\sigma^2,S,Q)$:


\[
    \mathcal{L}(\mu,\sigma^2,S,Q)
    = \log \ell(\mu,\sigma^2,S,Q)
    =
    \log\left[
        \left( \frac{1}{\sqrt{2\pi}\,\sigma} \right)^n\right]+ \log\left[
        \exp\!\left( -\frac{1}{2\sigma^2}
        (Q - 2\mu S + n\mu^2)
        \right)
        \right],
\]

\[
    \mathcal{L}(\mu,\sigma^2,S,Q)
    = -\frac{n}{2}\log(2\pi)
    -\frac{n}{2}\log(\sigma^2)
    -\frac{1}{2\sigma^2}\bigl(Q - 2\mu S + n\mu^2\bigr).
\]

%--------------------------- Derivada em relação a \mu

Calculando a derivada de $\mathcal{L}(\mu,\sigma^2,S,Q)$ em relação a $\mu$:
\[
    \frac{\partial \mathcal{L}(\mu,\sigma^2,S,Q)}{\partial\mu}
    = -\frac{1}{2\sigma^2}\,
    \frac{\partial}{\partial\mu}\left[
        Q - 2\mu S + n\mu^2\right]
    = -\frac{1}{2\sigma^2}\left(-2S + 2n\mu\right)= \frac{S - n\mu}{\sigma^2}.
\]

Seja a condição de máximo onde está definido $\hat\mu_{MV}$:

\[
    \frac{\partial \mathcal{L}(\mu,\sigma^2,S,Q)}{\partial\mu} = 0
    \;\Longleftrightarrow\;
    \frac{S - n\hat\mu_{MV}}{\sigma^2} = 0
    \;\Longleftrightarrow\;
    S - n\hat\mu_{MV} = 0.
\]


Isolando $\hat\mu_{MV}$:
\[\boxed{
        \widehat{\mu}_{MV} = \frac{S}{n}.}
\]

%%
%----------------------------------------------
% Derivada em relação a sigma^2
%----------------------------------------------

Calculando a derivada de $\mathcal{L}(\mu,\sigma^2,S,Q)$ em relação a $\sigma^2$:

\[
    \frac{\partial \mathcal{L}(\mu,\sigma^2,S,Q)}{\partial \sigma^2}
    =
    -\frac{n}{2}\,\frac{1}{\sigma^2}
    \;-\;
    \frac{\partial}{\partial \sigma^2}
    \left[
        \frac{1}{2\sigma^2}
        (Q - 2\mu S + n\mu^2)
        \right]
    = -\frac{n}{2\sigma^2}
    +
    \frac{1}{2}
    \frac{Q - 2\mu S + n\mu^2}{(\sigma^2)^2}.
\]

\[
    \frac{\partial \mathcal{L}(\mu,\sigma^2,S,Q)}{\partial \sigma^2}
    =
    \frac{ -n(\sigma^2) + (Q - 2\mu S + n\mu^2) }{2(\sigma^2)^2}.
\]

Seja a condição de máximo onde está definido $\hat\sigma_{MV}^2$:

\[
    \frac{\partial \mathcal{L}(\hat\mu_{MV},\sigma^2,S,Q))}{\partial \sigma^2} = 0
    \;\Longleftrightarrow\;
    -n(\hat\sigma_{MV}^2) + (Q - 2\hat\mu_{MV} S + n\hat\mu_{MV}^2) = 0.
\]

Isolando $\hat\sigma_{MV}^2$:

\[
    n\hat\sigma_{MV}^2 = Q - 2\hat\mu_{MV}S + n\hat\mu_{MV}^2 = Q - \frac{S^2}{n}, \qquad para \qquad \widehat{\mu}_{MV} = \frac{S}{n}.
\]

Assim,

\[
    n\widehat{\sigma}^2_{MV}
    = Q - \frac{S^2}{n}.
\]

Finalmente:

\[
    \boxed{
    \widehat{\sigma}^2_{MV}
    = \frac{1}{n}\left(Q - \frac{S^2}{n}\right)
    }
\]

\textbf{Estimador de máxima verossimilhança}

Substituindo de volta em funcão da variável aleatória $z_i$, obtemos:

\[
    \widehat\mu_{MV}
    = \frac{S}{n}
    = \frac{1}{n}\sum_{i=1}^n z_i
    = \overline Z
\]

\[
    \boxed{
        \widehat\mu_{MV}
        = \overline Z
    }
\]

\[
    \widehat\sigma^2_{MV}
    = \frac{1}{n}\left(Q - \frac{S^2}{n}\right)
    = \frac{1}{n}\left(\sum_{i=1}^n z_i^2
    - \frac{(n\bar{Z})^2}{n}\right)
    = \frac{1}{n}\left(\sum_{i=1}^n z_i^2
    - n\bar{Z}^2\right)
    = \frac{1}{n}\left(\sum_{i=1}^n z_i^2 - 2\bar{Z}n\bar{Z}
    + n\bar{Z}^2\right)
\]

\[
    \widehat\sigma^2_{MV} = \frac{1}{n}\!\left(\sum_{i=1}^n z_i^{2} - 2\overline Z \sum_{i=1}^n z_i + n\overline Z^{2}\right)
    = \frac{1}{n}\sum_{i=1}^n\!\left(z_i^{2}-2 z_i\overline Z+\overline Z^{2}\right)
    = \frac{1}{n}\sum_{i=1}^n (z_i - \overline Z)^2
\]

\[
    \boxed{
    \widehat\sigma^2_{MV} = \frac{1}{n}\sum_{i=1}^n (z_i - \overline Z)^2
    }
\]



%=========================================
% DESCARTAR - Derivação detalhada da densidade Normal e EMV
%=========================================

% \textbf{A distribuição Normal} surge como limite assintótico da soma de variáveis aleatórias independentes
% com variância finita, conforme estabelecido pelo \textbf{Teorema Central do Limite (TCL)}.

% \textbf{TCL:}
% Seja $\{X_1, X_2, \ldots, X_n\}$ uma sequência de variáveis aleatórias
% independentes e identicamente distribuídas (i.i.d.) com média $\mu = \mathbb{E}[X_i]$
% e variância $\sigma^2 = \mathrm{Var}(X_i) < \infty$.
% Definimos a média amostral
% \[
%     \bar{X}_n = \frac{1}{n}\sum_{i=1}^{n} X_i.
% \]

% Então, a variável associada à média amostral centralizada e reescalada para ter média 0 e variância 1
% \[
%     Z_n = \frac{\bar{X}_n - \mu}{\sigma / \sqrt{n}}
% \]
% converge em distribuição para uma Normal padrão, isto é,
% \[
%     Z_n \xrightarrow{d} \mathcal{N}(0,1)
%     \qquad \text{quando } n \to \infty.
% \]

% Em outras palavras, independentemente da distribuição original das $X_i$,
% a média amostral tende, para amostras grandes, a seguir aproximadamente uma
% \textbf{distribuição Normal} com média $\mu$ e variância $\sigma^2 / n$.

% ---

% Para modelar matematicamente a função densidade da distribuição Normal,
% buscamos uma função contínua, que satisfaça as condições de simetria e normalização, da forma:

% \[
%     f(z) = A \, e^{-k(z-\mu)^2},
% \]

% onde $A>0$ e $k>0$ são constantes a determinar. Essa escolha se justifica porque o termo $e^{-k(z-\mu)^2}$ é simétrico e decai rapidamente conforme $|z-\mu|$ aumenta, representando a concentração de probabilidade em torno de $\mu$.

% Impondo a condição de normalização $\int_{-\infty}^{+\infty} f(z)\,dz = 1$, temos:

% \[
%     A \int_{-\infty}^{+\infty} e^{-k(z-\mu)^2}\,dz = 1.
% \]

% Usando a mudança de variável \(x = \sqrt{k}\,(z-\mu)\), obtemos \(dz = dx/\sqrt{k}\), e assim:

% \[
%     A \int_{-\infty}^{+\infty} e^{-x^2}\frac{dx}{\sqrt{k}} = 1.
% \]

% Sabendo que \(\displaystyle \int_{-\infty}^{+\infty} e^{-x^2}\,dx = \sqrt{\pi}\), segue que

% \[
%     A = \sqrt{\frac{k}{\pi}}.
% \]

% Chamamos $\mu$ de \textbf{média} (ou valor esperado) e associamos $k$ à \textbf{variância} $\sigma^2$ por meio da relação

% \[
%     k = \frac{1}{2\sigma^2},
% \]

% que garante que $\mathrm{Var}(Z)=\sigma^2$. Substituindo $A$ e $k$, obtemos:

% \[
%     \boxed{
%         f(z\mid\mu,\sigma^2)
%         = \frac{1}{\sqrt{2\pi\sigma^2}}
%         \exp\!\left\{-\frac{(z-\mu)^2}{2\sigma^2}\right\}.
%     }
% \]

% Essa é a \textbf{função densidade de probabilidade} da distribuição Normal, denotada por $Z \sim \mathcal N(\mu,\sigma^2)$.

% ---

% Agora, considerando uma amostra aleatória \( (Z_1, \dots, Z_n) \) de \( Z \sim \mathcal N(\mu,\sigma^2) \), temos a função de densidade conjunta, a \textbf{função de verossimilhança}, como produtório das densidades individuais:

% \[
%     L(\mu,\sigma^2; z_1, \dots, z_n)
%     = \prod_{i=1}^n f(z_i\mid\mu,\sigma^2)
%     = (2\pi\sigma^2)^{-n/2}
%     \exp\!\left[-\frac{1}{2\sigma^2}\sum_{i=1}^n (z_i - \mu)^2\right].
% \]

% Calculando a função \textbf{log verossimilhança}, para simplificar a maximização:

% \[
%     \ell(\mu,\sigma^2)
%     = \log L(\mu,\sigma^2)
%     = -\frac{n}{2}\log(2\pi)
%     - \frac{n}{2}\log(\sigma^2)
%     - \frac{1}{2\sigma^2}\sum_{i=1}^n (z_i-\mu)^2.
% \]

% Sabendo que
% \(\dfrac{d}{d\mu}(z_i-\mu)^2 = -2(z_i-\mu)\)
% e que a derivada é linear em relação à soma, derivamos em relação a \textbf{$\mu$}:

% \[
%     \frac{\partial \ell}{\partial \mu}
%     = -\frac{1}{2\sigma^2}\sum_{i=1}^n [-2(z_i-\mu)]
%     = \frac{1}{\sigma^2}\sum_{i=1}^n (z_i-\mu).
% \]

% Igualando a zero para encontrar o ponto crítico:

% \[
%     \frac{\partial \ell}{\partial \mu}=0
%     \quad\Longleftrightarrow\quad
%     \sum_{i=1}^n (z_i-\mu)=0
%     \quad\Longleftrightarrow\quad
%     n\mu=\sum_{i=1}^n z_i.
% \]

% Assim, o \textbf{estimador de máxima verossimilhança} para $\mu$ é

% \[
%     \boxed{
%         \widehat\mu_{MV} = \bar{Z} = \frac{1}{n}\sum_{i=1}^n Z_i.
%     }
% \]

% ---

% Agora derivamos a \textbf{log-verossimilhança} em relação a \textbf{$\sigma^2$}:

% \[
%     \frac{\partial \ell}{\partial(\sigma^2)}
%     = -\frac{n}{2}\frac{1}{\sigma^2}
%     + \frac{1}{2(\sigma^2)^2}\sum_{i=1}^n (z_i-\mu)^2.
% \]

% Igualando a zero e substituindo $\mu = \widehat{\mu}_{MV}$:

% \[
%     0 = -\frac{n}{2\sigma^2}
%     + \frac{1}{2(\sigma^2)^2}\sum_{i=1}^n (z_i-\bar{Z})^2.
% \]

% Multiplicando ambos os lados por $2(\sigma^2)^2$ e rearranjando os termos:

% \[
%     n\sigma^2 = \sum_{i=1}^n (z_i-\bar{Z})^2
%     \quad\Longrightarrow\quad
%     \boxed{
%         \widehat{\sigma^2}_{MV} = \frac{1}{n}\sum_{i=1}^n (Z_i - \bar{Z})^2.
%     }
% \]

% ---

% \noindent
% Observação importante: O estimador de máxima verossimilhança da variância é \textit{viciado} para amostras finitas, pois subestima a verdadeira dispersão.
% Calculando a esperança para o estimador obtido, tem-se que:
% \[
%     E[\widehat{\sigma^2}_{MV}] = \frac{n-1}{n}\sigma^2.
% \]
% Esse viés ocorre porque a média amostral $\bar{Z}$ é calculada a partir dos próprios dados, introduzindo uma restrição linear entre os desvios $(Z_i - \bar{Z})$.
% Assim, apenas $n - 1$ deles são linearmente independentes, o que caracteriza a perda de um grau de liberdade.

% \[
%     \sum_{i=1}^{n}(Z_i - \bar{Z}) = 0 \quad \Longrightarrow \quad (Z_n - \bar{Z}) = -\sum_{i=1}^{n-1}(Z_i - \bar{Z}).
% \]

% Para eliminar o viés, multiplicamos o estimador por $\tfrac{n}{n-1}$,
% restaurando a estimativa não viciada da variância populacional:
% \[
%     S^2 = \frac{1}{n-1}\sum_{i=1}^{n}(Z_i - \bar{Z})^2,
%     \qquad \text{para o qual } \quad E[S^2] = \sigma^2.
% \]

% Essa correção garante que o estimador seja não viciado.
% Quando $n$ é grande, a diferença entre os estimadores é desprezível,
% e $\widehat{\sigma^2}_{MV}$ é assintoticamente não viciado e consistente.

% ---

% Derivando novamente para confirmar que o ponto crítico é de máximo:

% \[
%     \frac{\partial^2 \ell}{\partial \mu^2} = -\frac{n}{\sigma^2} < 0,
%     \qquad
%     \frac{\partial^2 \ell}{\partial(\sigma^2)^2}
%     = \frac{n}{2(\sigma^2)^2} - \frac{1}{(\sigma^2)^3}\sum_{i=1}^n (z_i-\mu)^2.
% \]

% Substituindo $\sum (z_i-\widehat\mu_{MV})^2 = n\widehat\sigma^2_{MV}$, obtemos valores negativos, confirmando o máximo.

% ---

% Os estimadores de máxima verossimilhança para os parâmetros da distribuição Normal são:

% \[
%     \boxed{
%         \widehat\mu_{MV} = \bar{Z},
%         \qquad
%         \widehat{\sigma^2}_{MV} = \frac{1}{n}\sum_{i=1}^n (Z_i - \bar{Z})^2.
%     }
% \]

\end{research}

\begin{answer}[]
    %------------------------------------------------------------
% Questão 03 – Itens (a)–(d)
%------------------------------------------------------------

\paragraph{(a) $g(\theta)=E_\theta(Z)$}
\[\]
Pela definição de esperança para variáveis contínuas, sendo $h(z)$ a função de interesse:
\[
  E_\theta(h(Z)) = \int_{-\infty}^{\infty} h(z)\, f(z;\theta)\,dz.
\]

Em particular, a esperança do primeiro momento de $Z$, é:
\[
  E_\theta(Z)
  = \int_{-\infty}^{\infty} z\,\frac{1}{\sqrt{2\pi\sigma^2}}
  e^{-\frac{(z-\mu)^2}{2\sigma^2}}\,dz
  = \int_{-\infty}^{\infty} (\mu + y)\,
  \frac{1}{\sqrt{2\pi\sigma^2}} e^{-\frac{y^2}{2\sigma^2}}\,dy
  = \mu,
\]

\[
  E_\theta(Z)
  = \mu \int_{-\infty}^{\infty}
  \frac{1}{\sqrt{2\pi\sigma^{2}}}\,e^{-\frac{y^{2}}{2\sigma^{2}}} \, dy
  \;+\;
  \int_{-\infty}^{\infty}
  y\,\frac{1}{\sqrt{2\pi\sigma^{2}}}\,e^{-\frac{y^{2}}{2\sigma^{2}}}\,dy
\]

\[
  E_\theta(Z)
  = \mu \int_{-\infty}^{\infty} \phi_\sigma(y)\,dy
  \;+\;
  \int_{-\infty}^{\infty} y\,\phi_\sigma(y)\,dy
  = \mu \cdot 1 + 0
  = \mu.
\]

Então:
\[
  g(\theta) = g(\mu,\sigma^2) = E_\theta(Z) = \mu.
\]

Pela invariância do EMV,
\[
  \widehat g_{MV} = g(\widehat\mu_{MV},\widehat\sigma^2_{MV})
  = \widehat\mu_{MV}
  = \overline Z.
\]

Portanto,
\[
  \boxed{\widehat g_{MV} = \overline Z.}
\]

%--------------------------- (b)

\paragraph{(b) $g(\theta)=P_\theta(Z<2)$}
\[\]

Seja a função distribuição acumulada da Normal padrão \(N(0,1)\) definida por:
\[
  \Phi(x)=\frac{1}{\sqrt{2\pi}}
  \int_{-\infty}^{x} e^{-t^{2}/2}\,dt.
\]

Integrando a função densidade de probabilidade $f(z_i;\mu,\sigma^2)$ para encontrar $P_\theta(Z<2)$, obtemos:
\[
  P_\theta(Z<2)
  = \int_{-\infty}^{2} \frac{1}{\sqrt{2\pi\sigma^2}}
  e^{-(z-\mu)^2/(2\sigma^2)}dz
  = \int_{-\infty}^{\frac{2-\mu}{\sqrt{\sigma^2}}}
  \frac{1}{\sqrt{2\pi}}\,
  e^{-y^{2}/2}\,dy
  = \Phi\!\left(\frac{2-\mu}{\sqrt{\sigma^2}}\right),
\]


Assim, expressamos o resultado em termos de \(\Phi\):
\[
  g(\theta) =
  g(\mu,\sigma^2)
  = \Phi\!\left(
  \frac{2-\mu}{\sqrt{\sigma^2}}
  \right).
\]

Pela \textbf{invariância do EMV}, o estimador de máxima verossimilhança de $g(\theta)$ é:
\[
  \widehat g_{MV}
  = g(\widehat\mu_{MV},\widehat\sigma^2_{MV})
  = \Phi\!\left(
  \frac{2-\widehat\mu_{MV}}{\sqrt{\widehat\sigma^2_{MV}}}
  \right)
  = \Phi\!\left(
  \frac{2-\overline Z}{\sqrt{\widehat\sigma^2_{MV}}}
  \right)
  = \Phi\!\left(
  \frac{(2-\overline Z)\sqrt n}{
      \sqrt{\sum_{i=1}^n (z_i-\overline Z)^2}
    }
  \right)
\]

\[
  \boxed{
    \widehat g_{MV}
    = \Phi\!\left(
    \frac{(2-\overline Z)\sqrt n}{
        \sqrt{\sum_{i=1}^n (z_i-\overline Z)^2}
      }
    \right)}
\]

%--------------------------- (c)

\paragraph{(c) $g(\theta)=P_\theta(2.6<Z<4)$}
\[\]
De forma similar, podemos decompor a probabilidade em termos da função de distribuição acumulada da Normal padrão \(N(0,1)\) :
\[
  P_\theta(2.6<Z<4)
  = P_\theta(Z<4) - P_\theta(Z\le 2.6)
  = \Phi\!\left(\frac{4-\mu}{\sqrt{\sigma^2}}\right)
  - \Phi\!\left(\frac{2.6-\mu}{\sqrt{\sigma^2}}\right).
\]

Logo,
\[
  g(\theta) = g(\mu,\sigma^2)
  = \Phi\!\left(\frac{4-\mu}{\sqrt{\sigma^2}}\right)
  - \Phi\!\left(\frac{2.6-\mu}{\sqrt{\sigma^2}}\right).
\]

Pela \textbf{invariância do EMV}, o estimador de máxima verossimilhança de \(g(\theta)\) é
\[
  \widehat g_{MV}
  = g(\widehat\mu_{MV},\widehat\sigma^2_{MV})
  = \Phi\!\left(
  \frac{4-\overline Z}{\sqrt{\widehat\sigma^2_{MV}}}
  \right)
  - \Phi\!\left(
  \frac{2.6-\overline Z}{\sqrt{\widehat\sigma^2_{MV}}}
  \right),
\]


\[
  \boxed{
    \widehat g_{MV}
    = \Phi\!\left(
    \frac{(4 - \overline Z)\sqrt n}{
        \sqrt{\sum_{i=1}^n (z_i - \overline Z)^2}
      }
    \right)
    -
    \Phi\!\left(
    \frac{(2.6 - \overline Z)\sqrt n}{
        \sqrt{\sum_{i=1}^n (z_i - \overline Z)^2}
      }
    \right)
  }.
\]

%--------------------------- (d)

\paragraph{(d) $g(\theta)=\mathrm{Var}_\theta(Z)$}
\[\]
A variância é encontrada por:

\[
  \mathrm{Var}(Z) = E_\theta(Z^2) - \bigl(E_\theta(Z)\bigr)^2.
\]

Pela definição de esperança para variáveis contínuas, sendo $h(z)$ a função de interesse:
\[
  E_\theta(h(Z)) = \int_{-\infty}^{\infty} h(z)\, f(z;\theta)\,dz.
\]

Em particular, a esperança do primeiro momento, e segundo momento de $Z$, para a distribuição Normal, são:

\[
  E_\theta(Z)
  = \mu.
\]

e

\[
  E_\theta(Z^2)
  = \int_{-\infty}^{\infty}
  z^{2} \,\frac{1}{\sqrt{2\pi\sigma^{2}}}
  \exp\!\left(-\frac{(z-\mu)^{2}}{2\sigma^{2}}\right)\,dz
  = \int_{-\infty}^{\infty}
  (\mu + \sigma y)^{2}\,
  \frac{1}{\sqrt{2\pi}}e^{-y^{2}/2}\,dy.
\]
\[
  E_\theta(Z^2)
  = \mu^{2}\int_{-\infty}^{\infty}
  \frac{1}{\sqrt{2\pi}}e^{-y^{2}/2}\,dy
  + 2\mu\sigma\int_{-\infty}^{\infty}
  y\,\frac{1}{\sqrt{2\pi}}e^{-y^{2}/2}\,dy
  + \sigma^{2}\int_{-\infty}^{\infty}
  y^{2}\,\frac{1}{\sqrt{2\pi}}e^{-y^{2}/2}\,dy.
\]
\[
  E_\theta(Z^2)
  = \mu^{2}\cdot 1
  + 2\mu\sigma\cdot 0
  + \sigma^{2}\cdot 1
  = \mu^{2} + \sigma^{2}.
\]


Portanto,
\[
  g(\theta)=g(\mu,\sigma^2) = \sigma^2.
\]

Pela \textbf{invariância do EMV}, o estimador de máxima verossimilhança de \(g(\theta)\) é:
\[
  \widehat g_{MV}
  = g(\widehat\mu_{MV},\widehat\sigma^2_{MV})
  = \widehat\sigma^2_{MV}
  = \frac{1}{n}\sum_{i=1}^n (z_i-\overline Z)^2.
\]

\[
  \boxed{
    \widehat g_{MV}
    = \frac{1}{n}\sum_{i=1}^n (z_i-\overline Z)^2.
  }
\]

\end{answer}

\textbf{Questão 04:}
Seja $(Z_1, \ldots, Z_n)$ uma amostra aleatória de $Z \sim f_\theta$, $\theta \in (0,\infty)$, tal que a função densidade de probabilidade é dada por
\[
    f_\theta(x) = \theta \, x^{\theta-1}, \quad x \in (0,1),
\]
e $f_\theta(x) = 0$, caso contrário.
\begin{itemize}
    \item[(a)] Encontre o EMV para $g(\theta) = E_\theta(Z)$.
    \item[(b)] Encontre o EMV para $g(\theta) = P_\theta(Z > 0.3)$.
    \item[(c)] Encontre o EMV para $g(\theta) = P_\theta(0 < Z < 0.1)$.
    \item[(d)] Encontre o EMV para $g(\theta) = \mathrm{Var}_\theta(Z)$.
    \item[(e)] Considere que os dados foram observados (0.12, 0.50, 0.20, 0.23, 0.30, 0.11). Encontre as estimativas de MV para os itens acima.
\end{itemize}

\begin{research}[]
    \paragraph{Distribuição Beta.}
A distribuição Beta é contínua no intervalo $(0,1)$, com densidade
\[
f(x;\alpha,\beta)
= \frac{1}{B(\alpha,\beta)}\, x^{\alpha-1}(1-x)^{\beta-1},
\quad
0<x<1,
\]
onde $B(\alpha,\beta)=\frac{\Gamma(\alpha)\Gamma(\beta)}{\Gamma(\alpha+\beta)}$ é a função Beta.
No caso particular $\mathrm{Beta}(\theta,1)$,
\[
f_\theta(x)=\theta\,x^{\theta-1},\quad 0<x<1,
\]
e a normalização é verificada por
\[
\int_0^1 \theta x^{\theta-1}\,dx = 1.
\]
Seu valor esperado e variância são
\[
E[X]=\frac{\theta}{\theta+1},
\qquad
\mathrm{Var}(X)=\frac{\theta}{(\theta+1)^2(\theta+2)}.
\]
O parâmetro $\theta$ controla a concentração:
valores $\theta<1$ favorecem $x$ próximos de $0$,
$\theta>1$ favorecem $x$ próximos de $1$,
e $\theta=1$ dá a distribuição uniforme.
\end{research}
\begin{answer}[]
    \paragraph{(a) $g(\theta)=\mathbb E_\theta(Z)$.}
\[
    \mathbb E_\theta(Z)=\frac{\theta}{\theta+1}
    \quad\Longrightarrow\quad
    \boxed{\ \widehat g_{(a)}=\frac{\widehat\theta_{MV}}{\widehat\theta_{MV}+1}\ }.
\]

\paragraph{(b) $g(\theta)=P_\theta(Z>0.3)$.}
\[
    P_\theta(Z>0.3)=1-P_\theta(0<Z\le0.3)=1-(0.3)^\theta
    \quad\Longrightarrow\quad
    \boxed{\ \widehat g_{(b)}=1-(0.3)^{\widehat\theta_{MV}}\ }.
\]

\paragraph{(c) $g(\theta)=P_\theta(0<Z<0.1)$.}
\[
    P_\theta(0<Z<0.1)=(0.1)^\theta
    \quad\Longrightarrow\quad
    \boxed{\ \widehat g_{(c)}=(0.1)^{\widehat\theta_{MV}}\ }.
\]

\paragraph{(d) $g(\theta)=\mathrm{Var}_\theta(Z)$.}
\[
    \mathrm{Var}_\theta(Z)=\frac{\theta}{(\theta+1)^2(\theta+2)}
    \quad\Longrightarrow\quad
    \boxed{\ \widehat g_{(d)}=\frac{\widehat\theta_{MV}}
        {(\widehat\theta_{MV}+1)^2(\widehat\theta_{MV}+2)}\ }.
\]

%---------------------------------------------------
\paragraph{(e) Sejam os dados observados: $(0.12,\,0.50,\,0.20,\,0.23,\,0.30,\,0.11)$}
\[\]
Com $n=6$,
\[
    \sum_{i=1}^n\log z_i
    =\log(0.12)+\log(0.50)+\log(0.20)+\log(0.23)+\log(0.30)+\log(0.11)
    \approx -9.3038.
\]
Logo,
\[
    \widehat\theta_{MV}
    =-\frac{6}{-9.3038}\approx 0.6449.
\]

Assim:

\begin{itemize}
    \item[(a)] $\displaystyle \widehat g_{MV}=\frac{\widehat\theta_{MV}}{1+\widehat\theta_{MV}}
              \approx \frac{0.6449}{1.6449}\approx 0.3921$

    \item[(b)] $\displaystyle \widehat g_{MV}=1-(0.3)^{\widehat\theta_{MV}}
              \approx 1-(0.3)^{0.6449}\approx 0.5400$

    \item[(c)] $\displaystyle \widehat g_{MV}=(0.1)^{\widehat\theta_{MV}}
              \approx (0.1)^{0.6449}\approx 0.2265$

    \item[(d)] $\displaystyle \widehat g_{MV}=\frac{\widehat\theta_{MV}}{(\widehat\theta_{MV}+1)^2(\widehat\theta_{MV}+2)}
              \approx \frac{0.6449}{(1.6449)^2(2.6449)}\approx 0.0901$
\end{itemize}

\medskip
\noindent
\textit{Resumo:}\;
$\widehat\theta_{MV}=-n/\sum\log Z_i\approx 0.6449$ e,
por invariância, as estimativas de (a)--(d) são os valores de $g(\theta)$
avaliados em $\widehat\theta$ como mostrado acima.


%%%%%%%%%%%%%%%%%%%%%%%%%%%%%%%%%%%%%%%%%%%%%%%%%%%%%%%%%%%%%%%%%%%%%%%%

%%%%%%%%%%%%%%%%%%%%%%%%%%%%%%%%%%%%%%%%%%%%%%%%%%%%%%%%%%%%%%%%%%%%%%%%

%%%%%%%%%%%%%%%%%%%%%%%%%%%%%%%%%%%%%%%%%%%%%%%%%%%%%%%%%%%%%%%%%%%%%%%%

%%%%%%%%%%%%%%%%%%%%%%%%%%%%%%%%%%%%%%%%%%%%%%%%%%%%%%%%%%%%%%%%%%%%%%%%
\[
    P_\theta(Z>0{,}3)
    =1 - 0{,}3^\theta.
\]

\[
    \boxed{
    \widehat g_{MV}^{(b)} = 1 - 0{,}3^{\,\widehat\theta_{MV}}
    }
\]

% ------------------------------------------------------------
\subsubsection*{(c) Probabilidade $P_\theta(0<Z<0{,}1)$}

\[
    P_\theta(0<Z<0{,}1) = 0{,}1^\theta.
\]

\[
    \boxed{
    \widehat g_{MV}^{(c)} = 0{,}1^{\,\widehat\theta_{MV}}
    }
\]

% ------------------------------------------------------------
\subsubsection*{(d) Variância}

Primeiro momento:

\[
    E_\theta(Z)=\frac{\theta}{\theta+1}.
\]

Segundo momento:

\[
    E_\theta(Z^2)=\frac{\theta}{\theta+2}.
\]

Logo,

\[
    \mathrm{Var}_\theta(Z)
    = \frac{\theta}{\theta+2}
    - \left(\frac{\theta}{\theta+1}\right)^2
    = \frac{\theta}{(\theta+1)^2(\theta+2)}.
\]

\[
    \boxed{
        \widehat g_{MV}^{(d)} =
        \frac{\widehat\theta_{MV}}{
            (\widehat\theta_{MV}+1)^2(\widehat\theta_{MV}+2)}
    }
\]


\[
    f_\theta(x)=\theta\,x^{\theta-1},\qquad x\in(0,1),\ \ \theta>0,
\]
e $f_\theta(x)=0$ caso contrário. (Trata-se de uma Beta$(\theta,1)$.)
%---------------------------------------------------
\paragraph{Momentos e probabilidades.}
Para $X\sim f_\theta$, para $a>- \theta$,
\[
    \mathbb E_\theta(X^a)=\int_0^1 x^a\,\theta x^{\theta-1}dx
    =\theta\int_0^1 x^{a+\theta-1}dx
    =\frac{\theta}{a+\theta}.
\]
Em particular,
\[
    \mathbb E_\theta(X)=\frac{\theta}{\theta+1},\qquad
    \mathbb E_\theta(X^2)=\frac{\theta}{\theta+2},\qquad
    \mathrm{Var}_\theta(X)=\frac{\theta}{(\theta+1)^2(\theta+2)}.
\]
Para $0<a<b\le1$,
\[
    P_\theta(a<X<b)=\int_a^b \theta x^{\theta-1}dx
    =\Big[x^\theta\Big]_a^b=b^\theta-a^\theta.
\]

%---------------------------------------------------

\end{answer}

\nocite{patriota2025notas}
\nocite{fonseca2008curso}
\bibliographystyle{plain}
\bibliography{refs}
\end{document}
\end{document}
