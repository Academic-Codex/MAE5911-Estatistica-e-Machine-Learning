\paragraph{Distribuição Beta.}
A distribuição Beta é contínua no intervalo $(0,1)$, com densidade
\[
f(x;\alpha,\beta)
= \frac{1}{B(\alpha,\beta)}\, x^{\alpha-1}(1-x)^{\beta-1},
\quad
0<x<1,
\]
onde $B(\alpha,\beta)=\frac{\Gamma(\alpha)\Gamma(\beta)}{\Gamma(\alpha+\beta)}$ é a função Beta.
No caso particular $\mathrm{Beta}(\theta,1)$,
\[
f_\theta(x)=\theta\,x^{\theta-1},\quad 0<x<1,
\]
e a normalização é verificada por
\[
\int_0^1 \theta x^{\theta-1}\,dx = 1.
\]
Seu valor esperado e variância são
\[
E[X]=\frac{\theta}{\theta+1},
\qquad
\mathrm{Var}(X)=\frac{\theta}{(\theta+1)^2(\theta+2)}.
\]
O parâmetro $\theta$ controla a concentração:
valores $\theta<1$ favorecem $x$ próximos de $0$,
$\theta>1$ favorecem $x$ próximos de $1$,
e $\theta=1$ dá a distribuição uniforme.