\paragraph{(a) $g(\theta)=E_\theta(Z)$}
\[\]
Pela definição de esperança para variáveis contínuas, sendo $h(z)$ a função de interesse:
\[
  E_\theta(h(Z)) = \int_{-\infty}^{\infty} h(z)\, f(z;\theta)\,dz.
\]

Em particular, a esperança do primeiro momento de $Z$, é:
\[
  E_\theta(Z)
  = \int_{-\infty}^{\infty} z\,\frac{1}{\sqrt{2\pi\sigma^2}}
  e^{-\frac{(z-\mu)^2}{2\sigma^2}}\,dz
  = \int_{-\infty}^{\infty} (\mu + y)\,
  \frac{1}{\sqrt{2\pi\sigma^2}} e^{-\frac{y^2}{2\sigma^2}}\,dy
  = \mu,
\]

\[
  E_\theta(Z)
  = \mu \int_{-\infty}^{\infty}
  \frac{1}{\sqrt{2\pi\sigma^{2}}}\,e^{-\frac{y^{2}}{2\sigma^{2}}} \, dy
  \;+\;
  \int_{-\infty}^{\infty}
  y\,\frac{1}{\sqrt{2\pi\sigma^{2}}}\,e^{-\frac{y^{2}}{2\sigma^{2}}}\,dy
\]

\[
  E_\theta(Z)
  = \mu \int_{-\infty}^{\infty} \phi_\sigma(y)\,dy
  \;+\;
  \int_{-\infty}^{\infty} y\,\phi_\sigma(y)\,dy
  = \mu \cdot 1 + 0
  = \mu.
\]

Então:
\[
  g(\theta) = g(\mu,\sigma^2) = E_\theta(Z) = \mu.
\]

Pela \textbf{invariância do EMV}, o estimador de máxima verossimilhança de \(g(\theta)\) é:
\[
  \widehat g_{MV} = g(\widehat\mu_{MV},\widehat\sigma^2_{MV})
  = \widehat\mu_{MV}
  = \overline Z.
\]

\[
  \boxed{\widehat g_{MV} = \overline Z.}
\]

%--------------------------- (b)

\paragraph{(b) $g(\theta)=P_\theta(Z<2)$}
\[\]

Seja a função distribuição acumulada da Normal padrão \(N(0,1)\) definida por:
\[
  \Phi(x)=\frac{1}{\sqrt{2\pi}}
  \int_{-\infty}^{x} e^{-t^{2}/2}\,dt.
\]

Integrando a função densidade de probabilidade $f(z_i;\mu,\sigma^2)$ para encontrar $P_\theta(Z<2)$, obtemos:
\[
  P_\theta(Z<2)
  = \int_{-\infty}^{2} \frac{1}{\sqrt{2\pi\sigma^2}}
  e^{-(z-\mu)^2/(2\sigma^2)}dz
  = \int_{-\infty}^{\frac{2-\mu}{\sqrt{\sigma^2}}}
  \frac{1}{\sqrt{2\pi}}\,
  e^{-y^{2}/2}\,dy
  = \Phi\!\left(\frac{2-\mu}{\sqrt{\sigma^2}}\right),
\]


Assim, expressamos o resultado em termos de \(\Phi\):
\[
  g(\theta) =
  g(\mu,\sigma^2)
  = \Phi\!\left(
  \frac{2-\mu}{\sqrt{\sigma^2}}
  \right).
\]

Pela \textbf{invariância do EMV}, o estimador de máxima verossimilhança de $g(\theta)$ é:
\[
  \widehat g_{MV}
  = g(\widehat\mu_{MV},\widehat\sigma^2_{MV})
  = \Phi\!\left(
  \frac{2-\widehat\mu_{MV}}{\sqrt{\widehat\sigma^2_{MV}}}
  \right)
  = \Phi\!\left(
  \frac{2-\overline Z}{\sqrt{\widehat\sigma^2_{MV}}}
  \right)
  = \Phi\!\left(
  \frac{(2-\overline Z)\sqrt n}{
      \sqrt{\sum_{i=1}^n (z_i-\overline Z)^2}
    }
  \right)
\]

\[
  \boxed{
    \widehat g_{MV}
    = \Phi\!\left(
    \frac{(2-\overline Z)\sqrt n}{
        \sqrt{\sum_{i=1}^n (z_i-\overline Z)^2}
      }
    \right)}
\]

%--------------------------- (c)

\paragraph{(c) $g(\theta)=P_\theta(2.6<Z<4)$}
\[\]
De forma similar, podemos decompor a probabilidade em termos da função de distribuição acumulada da Normal padrão \(N(0,1)\) :
\[
  P_\theta(2.6<Z<4)
  = P_\theta(Z<4) - P_\theta(Z\le 2.6)
  = \Phi\!\left(\frac{4-\mu}{\sqrt{\sigma^2}}\right)
  - \Phi\!\left(\frac{2.6-\mu}{\sqrt{\sigma^2}}\right).
\]

Logo,
\[
  g(\theta) = g(\mu,\sigma^2)
  = \Phi\!\left(\frac{4-\mu}{\sqrt{\sigma^2}}\right)
  - \Phi\!\left(\frac{2.6-\mu}{\sqrt{\sigma^2}}\right).
\]

Pela \textbf{invariância do EMV}, o estimador de máxima verossimilhança de \(g(\theta)\) é
\[
  \widehat g_{MV}
  = g(\widehat\mu_{MV},\widehat\sigma^2_{MV})
  = \Phi\!\left(
  \frac{4-\overline Z}{\sqrt{\widehat\sigma^2_{MV}}}
  \right)
  - \Phi\!\left(
  \frac{2.6-\overline Z}{\sqrt{\widehat\sigma^2_{MV}}}
  \right),
\]


\[
  \boxed{
    \widehat g_{MV}
    = \Phi\!\left(
    \frac{(4 - \overline Z)\sqrt n}{
        \sqrt{\sum_{i=1}^n (z_i - \overline Z)^2}
      }
    \right)
    -
    \Phi\!\left(
    \frac{(2.6 - \overline Z)\sqrt n}{
        \sqrt{\sum_{i=1}^n (z_i - \overline Z)^2}
      }
    \right)
  }.
\]

%--------------------------- (d)

\paragraph{(d) $g(\theta)=\mathrm{Var}_\theta(Z)$}
\[\]
A variância é definida por:

\[
  \mathrm{Var}(Z) = E_\theta(Z^2) - \bigl(E_\theta(Z)\bigr)^2.
\]

Pela definição de esperança para variáveis contínuas, sendo $h(z)$ a função de interesse:
\[
  E_\theta(h(Z)) = \int_{-\infty}^{\infty} h(z)\, f(z;\theta)\,dz.
\]

Em particular, a esperança do primeiro e segundo momentos de $Z$, para a distribuição Normal, são:

\[
  E_\theta(Z)
  = \mu.
\]

e

\[
  E_\theta(Z^2)
  = \int_{-\infty}^{\infty}
  z^{2} \,\frac{1}{\sqrt{2\pi\sigma^{2}}}
  \exp\!\left(-\frac{(z-\mu)^{2}}{2\sigma^{2}}\right)\,dz
  = \int_{-\infty}^{\infty}
  (\mu + \sigma y)^{2}\,
  \frac{1}{\sqrt{2\pi}}e^{-y^{2}/2}\,dy.
\]
\[
  E_\theta(Z^2)
  = \mu^{2}\int_{-\infty}^{\infty}
  \frac{1}{\sqrt{2\pi}}e^{-y^{2}/2}\,dy
  + 2\mu\sigma\int_{-\infty}^{\infty}
  y\,\frac{1}{\sqrt{2\pi}}e^{-y^{2}/2}\,dy
  + \sigma^{2}\int_{-\infty}^{\infty}
  y^{2}\,\frac{1}{\sqrt{2\pi}}e^{-y^{2}/2}\,dy.
\]
\[
  E_\theta(Z^2)
  = \mu^{2}\cdot 1
  + 2\mu\sigma\cdot 0
  + \sigma^{2}\cdot 1
  = \mu^{2} + \sigma^{2}.
\]

Então:

\[
  \mathrm{Var}_\theta(Z)
  = E(Z^{2}) - [E(Z)]^{2}
  = \mu^{2} + \sigma^{2} - \mu^{2}
  = \sigma^{2}.
\]

Portanto,
\[
  g(\theta)=g(\mu,\sigma^2) = \sigma^2.
\]

Pela \textbf{invariância do EMV}, o estimador de máxima verossimilhança de \(g(\theta)\) é:
\[
  \widehat g_{MV}
  = g(\widehat\mu_{MV},\widehat\sigma^2_{MV})
  = \widehat\sigma^2_{MV}
  = \frac{1}{n}\sum_{i=1}^n (z_i-\overline Z)^2.
\]

\[
  \boxed{
    \widehat g_{MV}
    = \frac{1}{n}\sum_{i=1}^n (z_i-\overline Z)^2.
  }
\]


\paragraph{(e) Sejam os dados observados: $(0{,}12,\,0{,}50,\,0{,}20,\,0{,}23,\,0{,}30,\,0{,}11)$}
\[\]
Para o cálculo da média: $\quad n = 6\quad$ e $ \quad\sum_{i=1}^n z_i = 0{,}12 + 0{,}50 + 0{,}20 + 0{,}23 + 0{,}30 + 0{,}11 = 1{,}46.\quad$ Portanto:
\[
  \widehat\mu_{MV} = \overline Z = \frac{1{,}46}{6} \approx 0{,}2433.
\]
\[
  \boxed{
    \widehat\mu_{MV} \approx 0{,}2433}
\]

Para a variância de máxima verossimilhança:
\[
  \widehat\sigma^2_{MV} = \frac{1}{n}\sum_{i=1}^n (z_i - \overline Z)^2
\]

\[
  \widehat\sigma^2_{MV} = (0{,}12 - 0{,}2433)^2 + (0{,}50 - 0{,}2433)^2 + (0{,}20 - 0{,}2433)^2 + (0{,}23 - 0{,}2433)^2 + (0{,}30 - 0{,}2433)^2 + (0{,}11 - 0{,}2433)^2
\]

\[
  \boxed{
  \widehat\sigma^2_{MV} \approx 0{,}01736}
  % \qquad
  % \widehat\sigma_{MV} \approx 0{,}1317.
\]



Por fim, o desvio-padrão de máxima verossimilhança é
\[
  \widehat\sigma_{MV}
  = \sqrt{\widehat\sigma^2_{MV}}
  \approx \sqrt{0{,}01736}
  \approx 0{,}1317.
\]
%
Assim:

\begin{itemize}
  \item[(a)] $\displaystyle \widehat g_{MV}
          = \widehat{E_\theta(Z)} = \widehat\mu_{MV}
          = \overline Z \approx 0{,}2433.$

  \item[(b)] $\displaystyle \widehat g_{MV}
          = \widehat{P_\theta(Z<2)}
          = \Phi\!\left(\frac{2 - \widehat\mu_{MV}}{\widehat\sigma_{MV}}\right)
          = \Phi\!\left(\frac{2 - \overline Z}{\widehat\sigma_{MV}}\right)
          \approx \Phi(13{,}33) \approx 1.$

  \item[(c)] $\displaystyle \widehat g_{MV}
          = \widehat{P_\theta(2{,}6 < Z < 4)}
          = \Phi\!\left(\frac{4 - \overline Z}{\widehat\sigma_{MV}}\right)
          - \Phi\!\left(\frac{2{,}6 - \overline Z}{\widehat\sigma_{MV}}\right)
          \approx \Phi(28{,}52) - \Phi(17{,}89) \approx 0.$

  \item[(d)] $\displaystyle \widehat g_{MV} = \widehat{\mathrm{Var}\theta(Z)} = \widehat\sigma^2_{MV} \approx 0{,}01736.$
\end{itemize}




\[\]